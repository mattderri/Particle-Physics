\documentclass[10.75pt,a4paper,openright,bottom=2cm]{article}
\usepackage[english]{babel}
\usepackage[T1]{fontenc} 
\usepackage[utf8]{inputenc}
\usepackage{graphicx}
\usepackage{auto-pst-pdf} 
\usepackage{float}
\usepackage{graphicx}
\usepackage{wrapfig}
\usepackage{subcaption}
\usepackage{textcomp}
\usepackage{geometry}
\usepackage{pdfpages}
\usepackage{amsmath}
\usepackage{amsfonts}
\usepackage{wrapfig}
\usepackage{lipsum} 
\usepackage{fancyhdr}
\usepackage{amsmath}
\usepackage{graphicx}
\usepackage{tcolorbox}
\usepackage{bbm}
\usepackage{braket}
\usepackage{amssymb}
\usepackage{pifont}
\newcommand{\cmark}{\ding{51}}%
\newcommand{\xmark}{\ding{55}}%
\usepackage[table]{xcolor, colortbl}
\usepackage{cancel}
\DeclareMathAlphabet{\pazocal}{OMS}{zplm}{m}{n}
\usepackage[colorlinks=true, allcolors=blue]{hyperref}
\usepackage{multicol}
\usepackage{physics}
\usepackage[super]{nth}
\usepackage{bbm}
\usepackage{tikz}
\usepackage[compat=1.1.0]{tikz-feynman}
\usetikzlibrary{positioning}
\usepackage{circuitikz}
\usetikzlibrary{arrows,shapes,positioning}
\usetikzlibrary {arrows.meta} 
\usetikzlibrary{angles,quotes}
\usepackage{subfig}
\newcommand{\beginbox}[1]{\begin{tcolorbox}[width=\textwidth,colback={yellow!50},title={#1},colbacktitle={gray!50},coltitle=black]}
\renewcommand{\endbox}{\end{tcolorbox}\noindent}
% \begin{tcolorbox}[width=\textwidth,colback={yellow!50},title={Rosenbluth Cross Section},colbacktitle={gray!50},coltitle=black]
\title{Particle Physics}
\author{Matteo D'Errigo}

\begin{document}
\maketitle
\tableofcontents
% \begin{abstract}
% \end{abstract}
\newpage
\section{The Static Quark Model}
\subsection{Quantum Numbers}
\begin{itemize}
    \item \textbf{Parity} $\mathbb{P}$:
    \[
    \mathbb{P}\ket{\Psi(q,\Vec{x},t)}=P\ket{\Psi(q,-\Vec{x},t)}
    \]
    Since $\mathbb{P}^2=\mathbbm{1}$, this implies that $P=\pm1$. The parity of the anti-particle is the opposite of the parity of the particle. Conventionally, parity +1 is assigned to quarks and leptons. For spin-0 bosons, particle and anti-particle have the same parity. The gauge bosons $\gamma$ and $G_\mu^{a=1,\cdots,8}$ have parity -1, the electroweak gauge bosons $W^\pm$ and $Z$ do not have a defined intrinsic parity since electroweak interaction does not conserve parity.
    \item \textbf{Charge conjugation} $\mathbb{C}$: it changes a particle p into an anti-particle $\overline{\text{p}}$.
    \[
    \mathbb{C}\ket{\text{p},\Psi(\Vec{x},t)}=C\ket{\overline{\text{p}},\Psi(\Vec{x},t)}
    \]
    Therefore, position, momentum and spin are unchanged while charge, baryon number, lepton number and strangeness are flipped. Its eigenvalues are $\pm1$, conserved in strong and electromagnetic interaction. Particles which are their own anti-particle are eigenstates of $\mathbb{C}$. $\mathbb{C}$-conservation is used in EM decays:
    \[
    \left\{
    \begin{aligned}
    &\pi^0\to\gamma\gamma: &&+1\to(-1)(-1) \quad \text{\cmark}\\
    &\pi^0\to\gamma\gamma\gamma: &&+1\to(-1)(-1)(-1) \quad \text{\xmark}
    \end{aligned}
    \right.
    \]
    \item \textbf{G-parity} $\mathbb{G}=\mathbb{C}\mathbb{R}_2$, where $\mathbb{R}_2$ is a rotation in the isospin space. Conserved only in strong interactions, producing selection rules.
\end{itemize}
\subsection{Hadrons}
Overt time, the concept of \textit{elementary particle} entered a deep crisis. The existence of many hadrons was seen as a contradiction with the elementary nature of the fundamental component of matter and it became natural to interpret hadrons as resonances of elementary components.\\
Gell-Mann and Ne'eman proposed a new classification, the \textbf{Eightfold Way} based on the symmetry group SU(3).
\begin{center}
\begin{tikzpicture}
  %\draw[very thin,color=gray] (-0.1,-1.1) grid (3.9,3.9);
%x+1/2y=-1-->y=-2-2x
  \draw[->] (0,0) -- (6,0) node[right] {$I_3$};
  \draw[->] (0,0) -- (0,6) node[above] {$Y$};
  \fill[color=red] (1,3) circle[radius=2pt] node[left] {$\pi^-$};
  \fill[color=red] (2,1) circle[radius=2pt] node[left] {$K^-$};
  \fill[color=red] (4,1) circle[radius=2pt] node[right] {$\overline{K}^0$};
  \fill[color=red] (2,5) circle[radius=2pt] node[left] {$K^0$};
  \fill[color=red] (4,5) circle[radius=2pt] node[right] {$K^+$};
  \fill[color=red] (5,3) circle[radius=2pt] node[right] {$\pi^+$};
  \fill[color=red] (3,3) circle[radius=2pt] node[above] {$\pi^0,\eta^0,\eta'$};
  \draw[color=black] (0.1,5) -- (-0.1,5) node[left] {$+1$};
  \draw[color=black] (0.1,3) -- (-0.1,3) node[left] {$0$};
  \draw[color=black] (0.1,1) -- (-0.1,1) node[left] {$-1$};
  \draw[color=black] (1,0.1) -- (1,-0.1) node[below] {$-1$};
  \draw[color=black] (3,0.1) -- (3,-0.1) node[below] {$0$};
  \draw[color=black] (5,0.1) -- (5,-0.1) node[below] {$+1$};
  \draw[color=blue] (4.5,0) -- (1.5,6)  node[right] {$Q=0$};
  \draw[color=blue] (2.5,0) -- (0.65,3.7)  node[above] {$Q=-1$};
  \draw[color=blue] (6,1) -- (3.5,6)  node[right] {$Q=+1$};
  \draw[color=red] (1,3) -- (2,1);
  \draw[color=red] (1,3) -- (2,5);
  \draw[color=red] (2,5) -- (4,5);
  \draw[color=red] (4,5) -- (5,3);
  \draw[color=red] (5,3) -- (4,1);
  \draw[color=red] (4,1) -- (2,1);
  %(y-0)/(4-0)=(x-2.5)/(0.5-2.5)-->(y-0)/4=(2.5-x)/2-->y=5-2x-->y=13-2x
  %\draw[color=blue] [domain=0:3.2] plot (\x,-2-2*\x) node[right] {$\nu$};
  % \draw[color=black] [domain=0:1.25] plot (\x,-\x+3);
  % \draw[color=red] [domain=1.25:2.8] plot(\x,-\x+4) node[right] {$\gamma$};
  % \draw[color=red] (1.25,1.75) -- (1.25, 2.75);
  % \draw[dashed, color=black] (1.25,1.75) -- (0,1.75) node[left] {$T_1\sim1$\,MeV};
  % \draw[dashed, color=black] (1.25,2.75) -- (0,2.75) node[left] {$T_2$};
  % \draw[dashed, color=black] (2.8, 1.2) -- (2.8,0) node[below] {today};
  % \draw[dashed, color=black] (2.8, 1.2) -- (0,1.2) node[left] {$T_{0,\gamma}=2.7\,\text{K}$};
  % \draw[dashed, color=black] (2.8, 0.2) -- (0,0.2) node[left] {$T_{0,\nu}=?$};
  % \x r means to convert '\x' from degrees to _r_adians:
  % \draw[color=blue]   plot (\x,{sin(\x r)})    node[right] {$f(x) = \sin x$};
  % \draw[color=orange] plot (\x,{0.05*exp(\x)}) node[right] {$f(x) = \frac{1}{20} \mathrm e^x$};
\end{tikzpicture}
\end{center}
Hadrons are classified in the $(I_3,Y)$ plane, where $I_3$ is the third component of the isospin and $Y$ is the strong hypercharge given by $Y=\pazocal{B}+S$, with $\pazocal{B}$ baryon number and $S$ strangeness. The strangeness enlarges the isospin symmetry group from SU(2) to SU(3). The Gell-Mann-Nishijima formula tells us that:
\begin{tcolorbox}[width=\textwidth,colback={yellow!50},title={Gell-Mann-Nishijima Formula},colbacktitle={gray!50},coltitle=black]
\[
Q=I_3+\frac{1}{2}Y
\]
\end{tcolorbox}
\noindent
This symmetry is called flavour SU(3), or SU(3)$_\text{F}$, to distinguish it from color SU(3), or SU(3)$_\text{C}$.\\
Particles form multiplets of SU(3)$_\text{F}$. Each multiplet contains particles with the same spin and intrinsic parity.\\
For \textbf{mesons}, we have an octet+singlet while for \textbf{baryons} octet+decuplet. Notice that for mesons, because of $\mathbb{CPT}$ the masses of the octet are symmetric with respect to $S=0$ and $I_3=0$ while for baryons the masses increase with $-S$.\\
\begin{minipage}{0.5\textwidth}
\begin{center}
\begin{tikzpicture}
  %\draw[very thin,color=gray] (-0.1,-1.1) grid (3.9,3.9);
%x+1/2y=-1-->y=-2-2x
  \draw[->] (0,0) -- (6,0) node[right] {$I_3$};
  \draw[->] (0,0) -- (0,6) node[above] {$Y$};
  \fill[color=red] (1,3) circle[radius=2pt] node[left] {$\Sigma^-$};
  \fill[color=red] (2,1) circle[radius=2pt] node[left] {$\Xi^-$};
  \fill[color=red] (4,1) circle[radius=2pt] node[right] {$\Xi^0$};
  \fill[color=red] (2,5) circle[radius=2pt] node[left] {$n$};
  \fill[color=red] (4,5) circle[radius=2pt] node[right] {$p$};
  \fill[color=red] (5,3) circle[radius=2pt] node[right] {$\Sigma^+$};
  \fill[color=red] (3,3) circle[radius=2pt] node[above] {$\Sigma^0,\Lambda$};
  \draw[color=black] (0.1,5) -- (-0.1,5) node[left] {$+1$};
  \draw[color=black] (0.1,3) -- (-0.1,3) node[left] {$0$};
  \draw[color=black] (0.1,1) -- (-0.1,1) node[left] {$-1$};
  \draw[color=black] (1,0.1) -- (1,-0.1) node[below] {$-1$};
  \draw[color=black] (3,0.1) -- (3,-0.1) node[below] {$0$};
  \draw[color=black] (5,0.1) -- (5,-0.1) node[below] {$+1$};
  \draw[color=blue] (4.5,0) -- (1.5,6)  node[right] {$Q=0$};
  \draw[color=blue] (2.5,0) -- (0.65,3.7)  node[above] {$Q=-1$};
  \draw[color=blue] (6,1) -- (3.5,6)  node[right] {$Q=+1$};
  \draw[color=red] (1,3) -- (2,1);
  \draw[color=red] (1,3) -- (2,5);
  \draw[color=red] (2,5) -- (4,5);
  \draw[color=red] (4,5) -- (5,3);
  \draw[color=red] (5,3) -- (4,1);
  \draw[color=red] (4,1) -- (2,1);
  %(y-0)/(4-0)=(x-2.5)/(0.5-2.5)-->(y-0)/4=(2.5-x)/2-->y=5-2x-->y=13-2x
  %\draw[color=blue] [domain=0:3.2] plot (\x,-2-2*\x) node[right] {$\nu$};
  % \draw[color=black] [domain=0:1.25] plot (\x,-\x+3);
  % \draw[color=red] [domain=1.25:2.8] plot(\x,-\x+4) node[right] {$\gamma$};
  % \draw[color=red] (1.25,1.75) -- (1.25, 2.75);
  % \draw[dashed, color=black] (1.25,1.75) -- (0,1.75) node[left] {$T_1\sim1$\,MeV};
  % \draw[dashed, color=black] (1.25,2.75) -- (0,2.75) node[left] {$T_2$};
  % \draw[dashed, color=black] (2.8, 1.2) -- (2.8,0) node[below] {today};
  % \draw[dashed, color=black] (2.8, 1.2) -- (0,1.2) node[left] {$T_{0,\gamma}=2.7\,\text{K}$};
  % \draw[dashed, color=black] (2.8, 0.2) -- (0,0.2) node[left] {$T_{0,\nu}=?$};
  % \x r means to convert '\x' from degrees to _r_adians:
  % \draw[color=blue]   plot (\x,{sin(\x r)})    node[right] {$f(x) = \sin x$};
  % \draw[color=orange] plot (\x,{0.05*exp(\x)}) node[right] {$f(x) = \frac{1}{20} \mathrm e^x$};
\end{tikzpicture}
\end{center}
\end{minipage}\hfill
\begin{minipage}{0.5\textwidth}
\begin{center}
\begin{tikzpicture}
  %\draw[very thin,color=gray] (-0.1,-1.1) grid (3.9,3.9);
%x+1/2y=-1-->y=-2-2x
  \draw[->] (0,0) -- (6,0) node[right] {$I_3$};
  \draw[->] (0,0) -- (0,8) node[above] {$Y$};
  \fill[color=red] (0.5,7) circle[radius=2pt] node[above] {$\Delta^-$};
  \fill[color=red] (2,7) circle[radius=2pt] node[above] {$\Delta^0$};
  \fill[color=red] (4,7) circle[radius=2pt] node[above] {$\Delta^+$};
  \fill[color=red] (5.5,7) circle[radius=2pt] node[above] {$\Delta^{++}$};
  \fill[color=red] (1.35,5) circle[radius=2pt] node[above right] {$\Sigma^{*-}$};
  \fill[color=red] (3,5) circle[radius=2pt] node[above] {$\Sigma^{*0}$};
  \fill[color=red] (4.65,5) circle[radius=2pt] node[above right] {$\Sigma^{*+}$};
  \fill[color=red] (2.2,3) circle[radius=2pt] node[above right] {$\Xi^{*+}$};
  \fill[color=red] (3.8,3) circle[radius=2pt] node[above right] {$\Xi^{*0}$};
  \fill[color=red] (3,1) circle[radius=2pt] node[below] {$\Omega^-$};
  \draw[color=black] (0.1,7) -- (-0.1,7) node[left] {$+1$};
  \draw[color=black] (0.1,5) -- (-0.1,5) node[left] {$0$};
  \draw[color=black] (0.1,3) -- (-0.1,3) node[left] {$-1$};
  \draw[color=black] (0.1,1) -- (-0.1,1) node[left] {$-2$};
  \draw[color=black] (1.35,0.1) -- (1.35,-0.1) node[below] {$-1$};
  \draw[color=black] (3,0.1) -- (3,-0.1) node[below] {$0$};
  \draw[color=black] (4.65,0.1) -- (4.65,-0.1) node[below] {$+1$};
  \draw[color=black] (5.5,7) -- (6,7) node[right] {$m\approx1232\,\text{MeV}$};
  \draw[color=black] (1.35,5) -- (6,5) node[right] {$m\approx1385\,\text{MeV}$};
  \draw[color=black] (2.2,3) -- (6,3) node[right] {$m\approx1533\,\text{MeV}$};
  \draw[color=black] (3,1) -- (6,1) node[right] {$m\approx1680\,\text{MeV}$};
  % \draw[color=blue] (4.5,0) -- (1.5,6)  node[right] {$Q=0$};
  % \draw[color=blue] (2.5,0) -- (0.65,3.7)  node[above] {$Q=-1$};
  % \draw[color=blue] (6,1) -- (3.5,6)  node[right] {$Q=+1$};
  \draw[color=red] (0.5,7) -- (5.5,7);
  \draw[color=red] (0.5,7) -- (3,1);
  \draw[color=red] (5.5,7) -- (3,1);
  %(y-1)/(7-1)=(x-3)/(5.5-3)-->(y-1)/7=(x-3)/2.5-->y-1=2.8x-8.4-->y=2.8x-7.4
  %5+7.4=12.4=2.8x-->x=4.43
  % \draw[color=red] (4,5) -- (5,3);
  % \draw[color=red] (5,3) -- (4,1);
  % \draw[color=red] (4,1) -- (2,1);
  %(y-0)/(4-0)=(x-2.5)/(0.5-2.5)-->(y-0)/4=(2.5-x)/2-->y=5-2x-->y=13-2x
  %\draw[color=blue] [domain=0:3.2] plot (\x,-2-2*\x) node[right] {$\nu$};
  % \draw[color=black] [domain=0:1.25] plot (\x,-\x+3);
  % \draw[color=red] [domain=1.25:2.8] plot(\x,-\x+4) node[right] {$\gamma$};
  % \draw[color=red] (1.25,1.75) -- (1.25, 2.75);
  % \draw[dashed, color=black] (1.25,1.75) -- (0,1.75) node[left] {$T_1\sim1$\,MeV};
  % \draw[dashed, color=black] (1.25,2.75) -- (0,2.75) node[left] {$T_2$};
  % \draw[dashed, color=black] (2.8, 1.2) -- (2.8,0) node[below] {today};
  % \draw[dashed, color=black] (2.8, 1.2) -- (0,1.2) node[left] {$T_{0,\gamma}=2.7\,\text{K}$};
  % \draw[dashed, color=black] (2.8, 0.2) -- (0,0.2) node[left] {$T_{0,\nu}=?$};
  % \x r means to convert '\x' from degrees to _r_adians:
  % \draw[color=blue]   plot (\x,{sin(\x r)})    node[right] {$f(x) = \sin x$};
  % \draw[color=orange] plot (\x,{0.05*exp(\x)}) node[right] {$f(x) = \frac{1}{20} \mathrm e^x$};
\end{tikzpicture}
\end{center}
\end{minipage}
When the Eightfold Way was proposed, only 9 members of the decuplet were known and the last one was initially predicted: it needed to have $Y=-2$ and $I_3=0$, hence $Q=-1$, $S=-3$ and $\pazocal{B}=1$. It was called $\Omega^-$ with a predicted mass of $m_{\Omega^-}\approx1680$\,MeV. This $\Omega^-$ can only decay to a $S=-2$ state: since EM and strong interaction conserve strangeness, the lightest $S$ and $\pazocal{B}$ conserving decay is:
\begin{align*}
\Omega^-&\to\Xi^0K^-\\
S:-3&\to-2-1\\
B:+1&\to+1+0
\end{align*}
However, this is impossible since $m_\Omega<m_\Xi+m_K$: it follows that it must decay via \textbf{S-violating weak interaction}.
\[
\Omega^-\to\Xi^0\pi^-/\Xi^-\pi^0/\Lambda^0K^-
\]
In 1964, Gell-Mann and Zweig proposed that all the hadrons are composed of three constituents, called \textbf{quarks}. At the time, there were only three quarks: up $u$, down $d$ and strange $s$.
\begin{table}[h]
    \centering
    \begin{tabular}{l|ccc}
    \hline
    \cellcolor{gray!50} & \cellcolor{yellow!50}$u$ & \cellcolor{yellow!50}$d$ & \cellcolor{yellow!50}$s$\\
    \hline\hline
    \cellcolor{yellow!50}Baryon number $\pazocal{B}$ & 1/3 & 1/3 & 1/3\\
    \hline
    \cellcolor{yellow!50}Spin $J$ & 1/2 & 1/2 & 1/2\\
    \hline
    \cellcolor{yellow!50}Isospin $I$ & 1/2 & 1/2 & 0\\
    \hline
    \cellcolor{yellow!50}\nth{3} Isospin $I_3$ & 1/2 & -1/2 & 0\\
    \hline
    \cellcolor{yellow!50}Strangeness $S$ & 0 & 0 & -1\\
    \hline
    \cellcolor{yellow!50}Hypercharge $Y=\pazocal{B}+S$ & 1/3 & 1/3 & -2/3\\
    \hline
    \cellcolor{yellow!50}$Q=I_3+\frac{1}{2}Y$ & 2/3 & -1/3 & -1/3\\
    \hline
    \end{tabular}
    \caption{Quantum numbers of the three quarks known at the time.}
    \label{3quak}
\end{table}\\
Baryons are combinations of three quarks $qqq$, anti-baryons are combinations of three anti-quarks $\overline{q}\overline{q}\overline{q}$. Mesons are pairs quarks+anti-quarks $q\overline{q}$ while anti-mesons are pairs anti-quarks+quarks: therefore, they are their own anti-particles.\\
Quarks form a triplet, which is the basic representation of SU(3): they can be represented in the $(I_3,Y)$ plane and their combinations, i.e. hadrons, are sums of such vectors. Mesons are bound states $q\overline{q}$, the states $\pi^+=u\overline{d}, \pi^-=d\overline{u}, K^+=u\overline{s}, K^0=d\overline{s}, K^-=s\overline{u}$ and $\overline{K}^0=s\overline{d}$ have no ambiguity but the combinations $u\overline{u}, d\overline{d}$ and $s\overline{s}$ have the same quantum numbers and these three states mix together. 
\subsection{SU(3)}
For SU(2) symmetry, the generators are the $2\times2$ \textbf{Pauli matrices}.\\
For SU(3) instead we have the $3\times3$ \textbf{Gell-Mann matrices}:
\[
\begin{aligned}
&\lambda_1 = \begin{pmatrix} 0 & +1 & 0 \\ +1 & 0 & 0 \\ 0 & 0 & 0 \end{pmatrix} &&\lambda_2 = \begin{pmatrix} 0 & -i & 0 \\ +i & 0 & 0 \\ 0 & 0 & 0 \end{pmatrix} &&\lambda_3 = \begin{pmatrix} +1 & 0 & 0 \\ 0 & -1 & 0 \\ 0 & 0 & 0 \end{pmatrix} &&\lambda_4 = \begin{pmatrix} 0 & 0 & +1 \\ 0 & 0 & 0 \\ +1 & 0 & 0 \end{pmatrix} \\
&\lambda_5 = \begin{pmatrix} 0 & 0 & -i \\ 0 & 0 & 0 \\ +i & 0 & 0 \end{pmatrix} &&\lambda_6 = \begin{pmatrix} 0 & 0 & 0 \\ 0 & 0 & +1 \\ 0 & +1 & 0 \end{pmatrix} &&\lambda_7 = \begin{pmatrix} 0 & 0 & 0 \\ 0 & 0 & -i \\ 0 & +i & 0 \end{pmatrix} &&\lambda_8 = \frac{1}{\sqrt{3}} \begin{pmatrix} +1 & 0 & 0 \\ 0 & +1 & 0 \\ 0 & 0 & -2 \end{pmatrix}
\end{aligned}
\]
The two diagonal matrices $\lambda_3$ and $\lambda_8$ are associated to the operators of $I_3$ and $Y$.\\
Define now $T_3:=\frac{1}{2}\lambda_3$ and $T_8:=\frac{1}{\sqrt{3}}\lambda_8$ and consider the following eigenvectors:
\[
\ket{u}=\begin{pmatrix}
    1 \\ 0 \\ 0
\end{pmatrix} \quad \ket{d}=\begin{pmatrix}
    0 \\ 1 \\ 0
\end{pmatrix} \quad \ket{s}=\begin{pmatrix}
    0 \\ 0 \\ 1
\end{pmatrix}
\]
By applying $T_3$ and $T_8$ to them, one obtains the correct quantum numbers.\\
Consider now the $\Delta^{++}$ resonance. The wave function associated is the product of spatial wave function, spin wave function and flavour wave function. $\Delta^{++}$ is the lightest $uuu$ state and by symmetry consideration we obtain that its wave function is symmetric. However, it is a fermion, so its wave function must be anti-symmetric. The solution to this problem was proposed by Greenberg, Han and Nambu by introducing a new quantum number for strongly interacting particles, the \textbf{colour}. The idea is that quarks exist in three colours, say red, green and blue. They sum like in a TV screen, so when they are all present the screen is white. The \textit{anti-color} is brought by anti-quarks and it is such that color+anti-color=white. Mesons and baryons are white and have no colour: they are a \textbf{colour singlet}. Therefore, when including colour in the wave function of $\Delta^{++}$ we recover its anti-symmetric nature.
\newpage
\section{Hadron Structure}
% \subsection{Fermi Gas Model}
% Nuclei are bound states of protons $p$ and neutrons $n$. In the Fermi gas model, they are assumed to be identical except for their charge: they are little spheres of radius $r=r_0$ and mass $m$, with spin 1/2 and bound inside the nucleus. We define with $N$ the number of neutrons, $Z$ the number of protons and $A$ as $A=N+Z$. The volume of the nucleus, is given by:
% \[
% V=\frac{4}{3}\pi r_0^3A
% \]
% Assuming there are no EM interaction but only nuclear ones, we have $N=Z=\frac{A}{2}$. Due to uncertainty principle, each proton/neutron fills a volume equal to $(2\pi\hbar)^3$. Protons and neutrons are confined in a well-shaped potential, being them fermions there are two protons/neutrons per energy level, with opposite spin $\Uparrow\Downarrow$. From these approximations, one can compute:
% \[
% n_{n,\Uparrow}=n_{n,\Downarrow}=n_{p,\Uparrow}=n_{p,\Downarrow}=\frac{N}{2}=\frac{Z}{2}=\frac{A}{4}=\frac{\frac{4}{3}\pi r_0^3A\frac{4}{3}\pi p_F^3}{(2\pi\hbar)^3}=\frac{2Ar_0^3p_F^3}{9\pi\hbar^3}\Rightarrow N=Z=\frac{A}{2}=\frac{4Ar_0^3p_F^3}{9\pi\hbar^3}
% \]
\subsection{Scattering Experiments}
Rutherford used $\alpha$ particles $(Z=2,A=4)$ as probes and gold as fixed target $(Z=79, A=197)$. The kinetic energy of the $\alpha$ particles was a few MeV and sometimes he observed a particle scattered by an angle $\theta>90^\circ$: this is impossible if matter is soft and homogeneous. The only explanation was that matter concentrates in small heavy bodies, the \textbf{nuclei}. How could he model this scattering? He tried with a two-body scattering, with only Coulomb interaction, non-relativistic and no QM. The key point is that the nucleus is small enough that the $\alpha$ particle always \textit{sees} its full charge. The matter is neutral overall, but the electrons are too light to deflect the $\alpha$ particle ($m_e/m_\alpha\approx1/8000$).\\ 
Energy and angular momentum are conserved, because of symmetry the only important thing is $\Delta p_y$:
\[
\Delta p=|\Vec{p}-\Vec{p}'|=2p\sin(\theta/2)=\Delta p_y=\int_{-\infty}^{+\infty}dtF_y=\int_{-\infty}^{+\infty}dt\frac{zZe^2}{4\pi\varepsilon_0}\frac{\cos(\beta)}{r^2(t)}
\]
where $\beta$ is the angle with respect to $\hat{y}$. The angular momentum $\Vec{L}$ is given by:
\[
\Vec{L}=pb=mr^2\frac{d\beta}{dt}\leftrightarrow dt=mr^2\frac{d\beta}{pb}
\]
here $b$ is the distance between the probe and the target. Substituting this in the previous expression gives us:
\[
\Delta p_y=2p\sin(\theta/2)=\int_{-(\pi-\theta)/2}^{+(\pi-\theta)/2}d\beta\frac{zZe^2}{4\pi\varepsilon_0}\frac{m\cos(\beta)}{pb}=\frac{zZe^2}{2\pi\varepsilon_0}\frac{m}{pb}\cos(\theta/2)
\]
From this, one obtains:
\[
\tan(\theta/2)=\frac{zZe^2}{4\pi\varepsilon_0}\frac{m}{p^2b}\Rightarrow db=-\frac{zZe^2}{4\pi\varepsilon_0}\frac{m}{p^2}\frac{d\theta}{2\sin^2(\theta/2)}
\]
We want $db$ because we need $d\sigma=2\pi bdb$:
\[
d\sigma=2\pi bdb=2\pi\left(\frac{zZe^2m}{4\pi\varepsilon_0p^2}\right)^2\frac{d\theta}{2\tan(\theta/2)\sin^2(\theta/2)}
\]
Being $d\Omega=2\pi\sin\theta d\theta=4\pi\sin(\theta/2)\cos(\theta/2)d\theta$, we get:
\[
\frac{d\sigma}{d\Omega}=\left(\frac{zZe^2m}{4\pi\varepsilon_0}\right)^2\frac{1}{4p^4\sin^4(\theta/2)}=\left(\frac{zZe^2m}{2\pi\varepsilon_0}\right)^2\frac{1}{|\Vec{p}-\Vec{p}'|^4}
\]
By defining $d_0:=\frac{zZe^2}{2\pi\varepsilon_0mv^2}$, it is possible to write:
\begin{tcolorbox}[width=\textwidth,colback={yellow!50},title={Rutherford Scattering},colbacktitle={gray!50},coltitle=black]
\[
\frac{d\sigma}{d\Omega}=\frac{d_0^2}{16\sin^4(\theta/2)}
\]
\end{tcolorbox}
\noindent
For a large $b$, $\theta$ is small: this sends $d\sigma/d\Omega$ to infinity.\\
If the $\alpha$ trajectory is external to the nucleus, it does not probe its internal structure, the Rutherford experiment could only limit $R_{\text{nucleus}}<10^{-14}$\,m. When the kinetic energy of the probe increases, $r_{\min}$ becomes smaller than $R_{\text{nucleus}}$: a Rutherford-like scattering on lead at fixed $\theta=60^\circ$ showed deviation for kinetic energy larger than 25 MeV, giving $r_{\min}=14$\,fm.\\
In scattering experiments, a probe, usually assumed point-like, hits a hadronic complex system. In the final state, the probe emerges unchanged while the nucleus may or may not survive intact. The underlying idea is to study the structure of the hadrons by observing the scattering.\\
We now look at the kinematics of this type of processes. The 4-momentum conservation tells us that:
\[
\left\{
\begin{aligned}
&\Vec{p}+\Vec{0}=\Vec{p}'+\Vec{p}_{\text{N}}\to\Vec{p}_{\text{N}}=\Vec{p}-\Vec{p}'\\
&E+M=E'+E_{\text{N}}\to E_{\text{N}}=E+M-E'
\end{aligned}
\right.
\]
It follows that:
\[
E_{\text{N}}^2-p_{\text{N}}^2=\cancel{M^2}=E^2+\cancel{M^2}+E'^2+2EM-2EE'-2ME'-p^2-p'^2+2pp'\cos\theta
\]
In the ultra-relativistic approximation, it is assumed that $p\approx E$ and $p'\approx E'$. Neglecting the mass of the probe (usually an electron), the expression above now becomes:
\[
0=\cancel{E^2+E'^2}+2EM-2EE'-2ME'\cancel{-E^2-E'^2}+2EE'\cos\theta\Rightarrow EM=E'[E(1-\cos\theta)+M]
\]
Finally, we get:
\[
E'=\frac{EM}{M+E(1-\cos\theta)}=\frac{EM}{M+2E\sin^2(\theta/2)}
\]
A useful quantity is the \textbf{momentum transfer} $\Vec{q}:=\Vec{p}-\Vec{p}'$. The relativistic equivalent is obtained considering the four-momenta:
\[
q^2=2m^2-2EE'+2pp'\cos\theta\approx-4EE'\sin^2(\theta/2):=-Q^2
\]
The relation for $E'$ becomes now:
\[
E'=\frac{EM}{M+\frac{Q^2}{2E'}}=\frac{2EME'}{2ME'+Q^2}\Rightarrow 2EM=2ME'+Q^2\to Q^2=2M(E-E')
\]
Kinematical limits tell us that for $\theta=0^\circ$ we have $E'=E$ and $Q^2=0$ while for $\theta=180^\circ$ $E'\approx0$ and $Q^2$ is maximum. Since $E'$ cannot be smaller than zero, $Q^2$ cannot be larger than $2ME$. The shaded region in the figure below represents this constraint.\\
\begin{center}
\begin{tikzpicture}
  %\draw[very thin,color=gray] (-0.1,-1.1) grid (3.9,3.9);
%x+1/2y=-1-->y=-2-2x
    \draw[fill=gray!10, color=gray!10] (5.3,5.3)--(5.3,0)--(6.5,0)--(6.5,5.3);
  \draw[->] (0,0) -- (6,0) node[below right] {$2M(E-E')$};
  \draw[->] (0,0) -- (0,6) node[above left] {$Q^2$};
  \draw[->, color=red] (0,0) -- (5,5);
  \draw [color=black] (5.25,0) -- (5.25,5.25);
  \draw [{Stealth}-, color=black] (5.25,3) -- (4.5,3) node[left] {$2ME$};
  \draw [{Stealth}-, color=black] (0,0) -- (1,3) node[above] {$E'=E$};
  \draw [{Stealth}-, color=red] (2.5,2.5) -- (2.5,1) node [below] {elastic scattering};
  \draw [{Stealth}-, color=black] (5.125,5.125) -- (4,5.125) node[left] {$E'\approx0$};
\end{tikzpicture}
\end{center}
The momentum transfer represents the scale of the scattering, structures smaller than $\sim1/|\Vec{q}|$ are not visible to the probe. For large values of $|\overline{q}|$ we have large $E$ but the opposite is not necessarily true: it is possible to have processes at high energy and large distance. The quest for smaller scales leads obviously to larger values of $Q^2$ and, therefore, larger energy.\\
In the \textbf{inelastic} case, e.g a process of the type $lN\to l'H$, kinematics is the same if the energies of the leptons $l$ and $l'$ is much larger than their masses. Despite this, we usually work with the following Lorentz-invariant variables:
\begin{tcolorbox}[width=\textwidth,colback={yellow!50},title={Lorentz-Invariant Variables},colbacktitle={gray!50},coltitle=black]
\[
\begin{aligned}
&\nu:=\frac{q\cdot P}{M}=E-E' &&\text{energy lost by the probe}\\
&x:=\frac{Q^2}{2M\nu} &&\text{fraction of the 4-momentum carried by the interacting parton}\\
&y:=\frac{q\cdot P}{p\cdot P}=\frac{\nu}{E} &&\text{fraction of the energy lost by the lepton}\\
&W^2:=M^2-Q^2+2M\nu &&\text{(mass)$^2$ of the hadron system in the final state}
\end{aligned}
\]
\end{tcolorbox}
\noindent
In the elastic case, $\nu$ and $Q^2$ are not independent:
\[
Q^2=2M(E-E')=2M\nu\Rightarrow x=1
\]
Therefore there is only one independent parameter, $E'$ or $\theta$, according to the measurements.\\
On the other hand, in the inelastic case we have:
\[
Q^2=M^2-W^2+2M\nu\le2M\nu\Rightarrow x\le1
\]
If $W$ is not fixed, $Q^2$ and $\nu$ are independent and we have two variables.\\
It is possible to redefine the kinematics of scattering processes in the plane $(2M\nu,Q^2)$, two Lorentz invariant variables usually used in the laboratory frame where the initial state hadron is at rest.\\
\begin{center}
\begin{tikzpicture}
  \draw[fill=gray!10, color=gray!10] (5.3,5.6)--(5.3,0)--(6.5,0)--(6.5,5.6);
  \draw[fill=gray!10, color=gray!10] (0,0)--(5.6,5.6)--(0,5.6);
  \draw[->] (0,0) -- (6,0) node[below right] {$2M\nu$};
  \draw[->] (0,0) -- (0,6) node[above left] {$Q^2$};
  \draw[color=red] (0,0) -- (5.25,5.25);
  \draw[color=blue] (0,0) -- (5.25,2.625);
  \draw [color=black] (5.25,0) -- (5.25,5.5);
  \draw [{Stealth}-, color=red] (2.5,2.5) -- (1.5,2.5) node [left] {$x=1$};
  \draw [{Stealth}-, color=blue] (3,1.5) -- (3,1) node [below] {$x<1$};
  \draw [{Stealth}-, color=black] (5.25,5.3) -- (4,5.3) node[left] {$\nu\approx E$};
\end{tikzpicture}
\end{center}
Again, the regions shaded in gray denotes the kinematically forbidden areas. The bisector $x=1$, i.e. $W^2=M^2$, defines the elastic scattering. For $x<1$ and at higher distance from the bisector we have the \textbf{deep inelastic scattering} (DIS) and, possibly, new physics.
\subsection{Electron-Nucleus Scattering}
In the '20s, QM entered the game. Rutherford formula still works in QM, using non-relativistic QM and Born approximation. However, the $\alpha$-nucleus scattering takes place between two nuclei so it is not suitable for measuring the nucleus structure. It was needed to replace the $\alpha$ with something point-like, an electron $e^-$. The dynamics of electron-nucleus scattering can be described by the Rutherford formula with an adjustment due to Mott:
\beginbox{Mott Cross Section}
\begin{align*}
\frac{d\sigma}{d\Omega}\Bigr|_{\substack{\text{Mott*}}}&=\frac{d\sigma}{d\Omega}\Bigr|_{\substack{\text{Rutherford}}}\left(1-\beta^2\sin^2(\theta/2)\right)\approx\frac{d\sigma}{d\Omega}\Bigr|_{\substack{\text{Rutherford}}}\cos^2(\theta/2)\\
&=\frac{4Z^2\alpha^2E'^2}{|\Vec{q}|^4}\cos^2(\theta/2)
\end{align*}
\endbox
The Mott* cross-section neglects the nucleus dimension and its recoil (that's why the *), but it takes into account the spin of the electron with the factor $\cos^2(\theta/2)$.\\
For relativistic particles, what is conserved is the \textbf{helicity} $h$, the projection of the spin along the momentum. 
\[
h=\frac{\Vec{s}\cdot\Vec{p}}{|\Vec{s}|\cdot|\Vec{p}|}
\]
This conservation requires the spin flip of the electron between initial and final state because the momentum also flips at $\theta=180^\circ$. The angular momentum is not conserved if the nucleus does not absorb the spin variation. Therefore, the scattering for $\theta\simeq180^\circ$ is forbidden.
\subsection{Form Factors}
Experiments agrees with Mott* cross section only for small $|\Vec{q}|$, otherwise the experimental cross section is smaller. Possibly, the reason of this disagreement is the structure of the nucleus which results in a smaller electric charge seen by the projectile. The \textbf{form factor} $\pazocal{F}(\Vec{q})$ is defined as the Fourier transform of the charge distribution $\rho(\Vec{x})=Zef(\Vec{x})$:
\[
\pazocal{F}(\Vec{q})=\int d^3xf(\Vec{x})\exp{i\frac{\Vec{q}\cdot\Vec{x}}{\hbar}}
\]
If $\rho(\Vec{x})$ depends only on $|\Vec{x}|$, then:
\[
\frac{d\sigma}{d\Omega}\Bigr|_{\substack{\text{Exp}}}=\frac{d\sigma}{d\Omega}\Bigr|_{\substack{\text{Mott*}}}|\pazocal{F}(q^2)|^2
\]
In principle, the function $\rho(r)$ may be computed by measuring $\pazocal{F}(q^2)$. However, the range of $q$ accessible to experiments is limited, therefore the behaviour of the form factor for large $q^2$ has to be extrapolated with reasonable assumptions.\\
Consider for example a homogeneous sphere with unit charge, $\rho(r)=\rho_0=\frac{3}{4\pi R^3}$ for $r\le R$ and 0 otherwise. The form factor can be computed as:
\begin{align*}
\pazocal{F}(q^2)&=4\pi\int_0^\infty dr\rho(r)r^2\frac{\sin(qr)}{qr}=\frac{4\pi\rho_0}{q}\int_0^Rdrr\sin(qr)=\frac{4\pi\rho_0}{q^3}\int_0^{t*}dtt\sin t\\
&=\frac{4\pi\rho_0}{q^3}[\sin(qR)-qR\cos(qR)]=\frac{3}{q^3R^3}[\sin(qR)-qR\cos(qR)]
\end{align*}
By comparing the first minimum of this function with the experiments for $^{12}$C ($q/\hbar\approx1.8$\,fm$^{-1}$), we get $R\approx4.5 r_{\min}\approx2.5$\,fm, i.e. $^{12}$C is approximately a sphere of radius 2.5 fm.\\
For $q\to0$, we can write:
\[
\pazocal{F}(q^2)\simeq1-\frac{1}{6}q^2\Braket{r^2}+\cdots \quad \Braket{r^2}:=\int d^3xf(\Vec{x})r^2
\]
Both the limit $q\to0$ and $q\to\infty$ have a deep meaning. $q$ is approximately the conjugate variable of the impact parameter $b$:
\begin{itemize}
    \item for very small $q$, i.e. $b$ very large, the target behaves as a point-like object
    \item for quite small $q$, the target behaves as a coherent homogeneous charged sphere with radius $\sqrt{\Braket{r^2}}$
    \item large $q$ probes the nucleus at small $b$. New physics requires very large $q$
\end{itemize}
Heavy nuclei are not homogeneous spheres with a sharp edge, rather spheres with a soft edge, with a charge distribution well reproduced by a Fermi function:
\[
\rho(r)=\frac{\rho_0}{1+\exp{\frac{r-c}{a}}}
\]
For large values of $A$, we get $c\approx1.07\,\text{fm}A^{1/3}$ and $a\approx0.54$\,fm.
Light nuclei, for example $^4$He, $^{6,7}$Li, $^9$Be, are more Gaussian-like.
\subsection{Nucleons Structure}
Probing smaller scales requires larger energies both in the initial and final state. For electron-nucleon scattering, take into account also the magnetic moment of the nucleons, by introducing the parameter $\tau=Q^2/4M^2$.\\
For point-like particles of mass $m$, charge $e$ and spin 1/2 the Dirac equation assigns them an intrinsic magnetic dipole moment: 
\[
\mu_C=\frac{ge\hbar}{4m}
\]
where $g$ is the \textbf{gyromagnetic ratio}. An ideal electron has $g=2$ and the first measurements roughly confirmed this value. This effect adds to the cross section a term corresponding to the spin flip probability.
\[
\frac{d\sigma}{d\Omega}\Bigr|_{\substack{\text{spin 1/2}}}=\frac{d\sigma}{d\Omega}\Bigr|_{\substack{\text{Mott}}}\left(1+2\frac{Q^2}{4M^2}\tan^2(\theta/2)\right)=\frac{d\sigma}{d\Omega}\Bigr|_{\substack{\text{Mott}}}\left(1+2\tau\tan^2(\theta/2)\right)
\]
The spin flip is particularly relevant for large $Q^2$ and large $\theta$.\\
\beginbox{Scattering Summary}
\begin{itemize}
    \item Rutherford, no electron spin, no nucleus recoil and no magnetic moment:
    \[
    \frac{d\sigma}{d\Omega}\Bigr|_{\substack{\text{Rutherford}}}=\frac{4Z^2\alpha^2E'^2}{|\Vec{q}|^4}
    \]
    \item Mott*, considering the electron spin:
    \[
    \frac{d\sigma}{d\Omega}\Bigr|_{\substack{\text{Mott*}}}=\frac{d\sigma}{d\Omega}\Bigr|_{\substack{\text{Rutherford}}}\left(1-\beta^2\sin^2(\theta/2)\right)
    \]
    \item Mott, considering the nucleus recoil:
    \[
    \frac{d\sigma}{d\Omega}\Bigr|_{\substack{\text{Mott}}}=\frac{d\sigma}{d\Omega}\Bigr|_{\substack{\text{Mott*}}}\frac{E'}{E}
    \]
    \item Adding the magnetic moment:
    \[
    \frac{d\sigma}{d\Omega}\Bigr|_{\substack{\text{spin 1/2}}}=\frac{d\sigma}{d\Omega}\Bigr|_{\substack{\text{Mott}}}\left(1+2\frac{Q^2}{4M^2}\tan^2(\theta/2)\right)
    \]
\end{itemize}
\endbox
For the nucleons, we define:
\[
\mu_{\text{N}}=\frac{e\hbar}{4m_{\text{N}}}
\]
If protons and neutrons were ideal Dirac particles, they should have:
\[
\mu_p=2\mu_{\text{N}} \quad \mu_n=0\longleftrightarrow\frac{g_p}{2}=\frac{\mu_p}{\mu_{\text{N}}}=1 \quad \frac{g_n}{2}=0
\]
Instead, experimental measurements disagree with these values: there are other effects which contribute to the magnetic moment, $p$ and $n$ are not ideal Dirac particles, \textit{maybe} they are not point-like.\\
In an electron-nucleon scattering, the main contribution is from single photon exchange.\\
\begin{center}
\begin{tikzpicture}
  \begin{feynman}
    \vertex [dot] (v1) {};
    \vertex [dot, below=of v1] (v2) {};
    \vertex [left=of v1] (e1) {$e^-$};
    \vertex [above right=of v1] (e2) {$e^-$};
    \vertex [right=of v1] (a);
    \vertex [left=of v2] (N1) {$N$};
    \vertex [right=of v2] (N2) {$N$};
    \diagram* {
    (v1) -- [photon, edge label=$\gamma^*$] (v2),
    (e1) -- [fermion] (v1),
    (v1) -- [fermion] (e2),
    (v1) -- [scalar] (a),
    (N1) -- [fermion] (v2),
    (v2) -- [fermion] (N2),
    };
    \draw pic[draw,angle radius=0.6cm,"$\theta$" shift={(4mm,2mm)}] {angle=a--v1--e2};
  \end{feynman}
\end{tikzpicture}
\end{center}   
The $ee\gamma^*$ vertex is well understood, while the $NN'\gamma^*$ is unknown. The strategy is to assume a simpler process, compare it with experiments and opportunely modify the theory inserting parameters which model the nucleon structure.\\
The cross section is generalized by defining the \textbf{Rosenbluth cross section}, introducing two form factors $G_{\text{E}}(Q^2)$ for the electric part and $G_{\text{M}}(Q^2)$ for the magnetic part.
\begin{tcolorbox}[width=\textwidth,colback={yellow!50},title={Rosenbluth Cross Section},colbacktitle={gray!50},coltitle=black]
\[
\frac{d\sigma}{d\Omega}\Bigr|_{\substack{\text{Rosenbluth}}}=\frac{d\sigma}{d\Omega}\Bigr|_{\substack{\text{Mott}}}\left(\frac{G_{\text{E}}^2+\tau G_{\text{M}}^2}{1+\tau}+2\tau G_{\text{M}}^2\tan^2(\theta/2)\right)
\]
\end{tcolorbox}
\noindent
For a charged Dirac fermion $f$, a proton $p$ and a neutron $n$ we have:
\[
\begin{aligned}
&f: &&G_{\text{E}}(\text{any $Q^2$})=1 &&G_{\text{M}}(\text{any $Q^2$})=1\\
&p: &&G_{\text{E}}(Q^2=0)=1 &&G_{\text{M}}(Q^2=0)\approx2.79\\
&n: &&G_{\text{E}}(Q^2=0)=0 &&G_{\text{M}}(Q^2=0)\approx-1.91
\end{aligned}
\]
In 1956, the Hofstadter spectrometer measured the elastic process $ep\to ep$. It measured $\theta$ in the range $[35^\circ,138^\circ]$ and therefore $Q^2$.
\[
E'=\frac{E}{1+\frac{E}{M}(1-\cos\theta)}\longleftrightarrow Q^2=2EE'(1-\cos\theta)
\]
At small $\theta$, i.e. small $Q^2$, all formulas agree, $G_{\text{E}}(Q^2\simeq0)\approx1$. However, at large $\theta$, i.e. large $Q^2$, the results disagree with any theoretical prediction.\\ Consider the Rosenbluth formula at fixed $Q^2$:
\[
\frac{d\sigma}{d\Omega}\Bigr|_{\substack{\text{Rosenbluth}}}\Bigr/\frac{d\sigma}{d\Omega}\Bigr|_{\substack{\text{Mott}}}=\left({\color{blue}\frac{G_{\text{E}}^2+\tau G_{\text{M}}^2}{1+\tau}}+{\color{green}2\tau G_{\text{M}}^2}\tan^2(\theta/2)\right)={\color{blue}A}+{\color{green}B}\tan^2(\theta/2)
\]
It is possible to measure $A$ and $B$ from which one gets $G_{\text{E}}$ and $G_{\text{M}}$ for both protons and neutrons. By repeating this process varying $Q^2$, it is possible to obtain the full dependence on the form factors. Results show that electric and magnetic form factors tend to a universal function of $Q^2$: the nucleons do not look like point-like particles or homogeneous spheres but like \textbf{diffused non-homogeneous systems}.\\
For $Q^2\ll m_p$, hence small $\tau$, G$_{\text{E}}$ dominates the cross section. In this range, we measure the average radius of electric charge, $\Braket{r_{\text{E}}}=0.85\pm0.02$\,fm.\\
In the range $0.02\le Q^2\le3$\, GeV$^2$, G$_{\text{E}}$ and G$_{\text{M}}$ are equally important while for $Q^2>3$\,GeV$^2$, the dominant contribution comes from G$_{\text{M}}$. Notice that if the proton was point-like, one would find both form factors equal to 1 and independent on $Q^2$.\\
At this level, it is unclear whether the nucleons have substructures, experiments at larger $Q^2$ are required: at higher $Q^2$, one can expect a variety of phenomena. 
\subsection{Deep Inelastic Scattering}
The usual parametrization of the cross section in the Deep Inelastic Scattering (DIS) region is:
\begin{align*}
\frac{d^2\sigma}{d\Omega dE'}\Bigr|_{\substack{\text{DIS}}}&=\frac{d\sigma}{d\Omega}\Bigr|_{\substack{\text{Mott}}}[W_2(Q^2,\nu)+2W_1(Q^2,\nu)\tan^2(\theta/2)]\\
&=\frac{4Z^2\alpha^2\hbar^2c^2E'^2}{|qc|^4}\cos^2(\theta/2)[W_2(Q^2,\nu)+2W_1(Q^2,\nu)\tan^2(\theta/2)]\\
&=\frac{4\alpha^2E'^2}{Q^4}[W_2(Q^2,\nu)\cos^2(\theta/2)+2W_1(Q^2,\nu)\sin^2(\theta/2)]
\end{align*}
The inelastic cross section requires two final-state variables, since $Q^2$ and $\nu$ are Lorentz-invariant they are more convenient. $W_1$ and $W_2$ are combinations of $G_{\text{E}}$ and $G_{\text{M}}$:
\beginbox{Structure Functions}
\[
W_1(Q^2,\nu)=\frac{F_1(x,Q^2)}{M}=\frac{3}{E}\tau G^2_{\text{M}} \quad W_2(Q^2,\nu)=\frac{F_2(x,Q^2)}{\nu}=\frac{3}{E}\left(\frac{G^2_{\text{E}}+\tau G_{\text{M}}^2}{1+\tau}\right)
\]
\endbox
They are known as \textbf{structure functions} and must be measured. The dynamics of the scattering depend on the structure of the target: $W_1$ and $W_2$ contain this information. Notice that there is no deep difference between $W_{1,2}$ and $F_{1,2}$, just use the more convenient ones.\\
From results, one observes that the structure functions appear to be nearly independent of $Q^2$. On the other hand, the elastic scattering for a non-point-like target has a strong dependence on $Q^2$. This means that for DIS the target behaves like a \textbf{point-like particle}: this $Q^2$ independence is another confirmation that DIS \textit{breaks} the proton, the scattering happens with its constituents.\\
Consider now the cross section of spin 1/2 particle of mass $m$, à la Rosenbluth with $G_{\text{E}}=G_{\text{M}}=1$, i.e. point-like particles:
\[
\frac{d\sigma}{d\Omega}\Bigr|_{\substack{\text{Rosenbluth}}}=\frac{d\sigma}{d\Omega}\Bigr|_{\substack{\text{Mott}}}(1+2\tau\tan^2(\theta/2))
\]
Take now the ratio of the magnetic and electric component both in the Rosenbluth cross section and in the DIS cross section:
\[
\begin{aligned}
&\text{DIS}: &&\frac{2W_1(Q^2,\nu)\sin^2(\theta/2)}{W_2(Q^2,\nu)\cos^2(\theta/2)}=\frac{2\nu F_1(x,Q^2)\tan^2(\theta/2)}{MF_2(x,Q^2)}\\
&\text{Rosenbluth}: &&2\tau\tan^2(\theta/2)=\frac{Q^2}{2M^2}\tan^2(\theta/2)=\frac{2\nu^2}{Q^2}\tan^2(\theta/2)
\end{aligned}
\]
where in the last step for the Rosenbluth part we used the fact that for the elastic scattering case $x=1$, hence $Q^2=2M\nu$. We can finally set these two relations equal to each other:
\[
\frac{2\nu F_1(x,Q^2)\tan^2(\theta/2)}{MF_2(x,Q^2)}=\frac{2\nu^2}{Q^2}\tan^2(\theta/2)\to\frac{F_1(x,Q^2)}{MF_2(x,Q^2)}=\frac{\nu}{Q^2}=\frac{1}{2Mx}\Rightarrow F_2(x,Q^2)=2xF_1(x,Q^2)
\]
This is known as the \textbf{Callan-Gross relation}, $F_2(x)=2xF_1(x)$.
\subsection{Parton Model}
Assume now that nucleons are made up of \textbf{partons}, later identified with \textbf{quarks}. Partons are spin 1/2 point-like fermions with a certain mass $m_i$  which scatter elastically in the $ep$ process. The DIS cross section reduces to an incoherent sum of constituents cross sections.
\[
\frac{d^2\sigma}{d\Omega dE'}\Bigr|_{\substack{m_i}}=\frac{4\alpha^2E'^2}{Q^4}\sum_i\left[e_i^2\left(\cos^2(\theta/2)+\frac{Q^2}{2m_i^2}\sin^2(\theta/2)\right)\delta\left(\nu-\frac{Q^2}{2m_i}\right)\right]
\]
where the charge of each of them are $qe_i$ and the delta function means that, at the constituent level, the scattering is elastic.\\
For such partons, we have:
\[
F_1(x)=MW_1(Q^2,\nu)=\frac{1}{2}\sum_je_j^2f_j(x) \quad F_2(x)=\nu W_2(Q^2,\nu)=x\sum_je_j^2f_j(x)
\]
They do not depend on $Q^2$ and $\nu$ separately but only on their ratio. This mechanism called \textbf{Bjorken scaling} was interpreted by Feynman in 1969 as the dominance of partons in the nucleon dynamics. 
Define now:
\[
x_{\text{F}}=\frac{\Vec{p}_{\text{parton}}}{\Vec{p}_{\text{nucleon}}}=x_{\text{B}}=x=\frac{Q^2}{2M\nu}=\frac{m}{M}
\]
The interaction electron-parton is so fast and violent that they behave like free particles, in \nth{1} approximation the other partons do not take part in the interaction.\\
This model implies that $\sum_ix_i=1$ when the sum runs over all partons. At the time, there was no clue about the nature of the partons, nor if they were charged or neutral, therefore $\sum_i'x_i\le1$ when the sum runs over the partons which interacts with the electron. If partons are spin 1/2 particles, then the Callan-Gross relation holds, instead if they have spin 0, $F_1(x)=0$. At the end, they are measured to have spin 1/2.\\
What is the dynamical meaning of $F_{1,2}$? In principle, proton and neutron have different structure functions. However, these functions are not independent: if they parametrize the actual structure of nucleons, they must be correlated. Assume now that the nucleons are made by three quarks, each of them described by a $x$ distribution identified with $f_j^p(x)$ (e.g. $u^p(x)$ denotes the $x$ distribution of up quarks in the proton, with $x$ in the interval $(x,x+dx)$).\\
For short intervals, QM allows quark-anti-quark pairs to exist in the nucleons. Moreover, in the hadrons we have some neutral particles, called \textbf{gluons}. Therefore, in the nucleons there are three types of particles.
\beginbox{Particles in the Nucleons}
\begin{itemize}
    \item \textbf{Valence quarks} with distribution $q_V(x)$, e.g. $u_V^p(x)$
    \item \textbf{Sea quarks}, i.e. the quark-anti-quark pairs, described by distributions $q_S(x)$, e.g. $u_S^p(x)$
    \item \textbf{Gluons} described by distributions $g^p(x)$ and $g^n(x)$
\end{itemize}
\endbox
What can be measured are sums like $u^p(x):=u^p_V(x)+u^p_S(x)$ and so on.\\ 
From the Callan-Gross relation we know that:
\[
F_2(x)=2xF_1(x)=x\sum_je_j^2f_j(x)
\]
Putting everything together and neglecting heavy quarks we obtain:
\begin{align*}
F_2^{\text{ep}}(x)&=x\left\{\frac{4}{9}[u^p(x)+\overline{u}^p(x)]+\frac{1}{9}[d^p(x)+\overline{d}^p(x)]+\frac{1}{9}[s^p(x)+\overline{s}^p(x)]\right\}\\
&=x\left\{\frac{4}{9}[u_V(x)+2q_S(x)]+\frac{1}{9}[d_V(x)+2q_S(x)]+\frac{1}{9}[2q_S(x)]\right\}=x\left\{\frac{4}{9}u_V(x)+\frac{1}{9}d_V(x)+\frac{4}{3}q_S(x)\right\}\\
F_2^{\text{en}}(x)&=x\left\{\frac{1}{9}u_V(x)+\frac{4}{9}d_V(x)+\frac{4}{3}q_S(x)\right\}
\end{align*}
Then, it follows that:
\[
\frac{F_2^{\text{en}}}{F_2^{\text{ep}}}=R_{\text{np}}=\left\{\begin{aligned}&1 &&(a)\\
&\frac{4d_V(x)+u_V(x)}{4u_V(x)+d_V(x)} &&(b)\end{aligned}\right.
\]
(a) corresponds to the case in which sea dominates, while (b) corresponds to the case in which valence dominates. Measurements show that (a) happens at low $x$ while (b) dominates at high $x$. In other words, there are plenty of $q\overline{q}$ pairs at low momentum, while valence is important at high $x$. Moreover, one obtains:
\[
F_2^{\text{ep}}-F_2^{\text{en}}=\frac{1}{3}x[u_V(x)-d_V(x)]
\]
And if, from the naive quark model, $u_V(x)\approx2d_V(x)$ we get $F_2^{\text{ep}}-F_2^{\text{en}}=\frac{1}{3}xd_V(x)$.\\
The integrals of $F_2(x)$ are calculable and measurable. By neglecting the small contribution of $s\overline{s}$, it is possible to obtain:
\begin{align*}
&\int_0^1dxF_2^{\text{ep}}(x)=\frac{4}{9}f_u+\frac{1}{9}f_d\\
&\int_0^1dxF_2^{\text{en}}(x)=\frac{1}{9}f_u+\frac{4}{9}f_d
\end{align*}
where $f_{u,d}$ are the fractions of proton and neutron momentum carried by the quark $u,d$(+ their respective anti-quarks). From direct measurements, we get:
\begin{align*}
&\int_0^1dxF_2^{\text{ep}}=\frac{4}{9}f_u+\frac{1}{9}f_d\approx0.18\\
&\int_0^1dxF_2^{\text{en}}=\frac{1}{9}f_u+\frac{4}{9}f_d\approx0.12
\end{align*}
Hence it follows $f_u\approx0.36$ and $f_d\approx0.18$. Note that $f_u+f_d\approx0.54$: only roughly a half of the nucleon momentum is carried by quarks and anti-quarks, the rest is \textit{invisible} in the DIS by a charged lepton. This was one of the first evidences for the existence of gluons, they are neutral and do not \textit{see} the EM interactions.
% Modern experiments have probed the nucleus to very high values of $Q^2$, up to $Q^2\approx10^5$\,Gev$^2$. When plotting $F_2$ as a function of $Q^2$ at fixed $x$, some $Q^2$ dependence appears, incompatible with Bjorken scaling. However, this effect is not attributed to sub-structures or other new physics, but to a dynamical change in $F_2$ well understood in QCD.
\newpage
\section{Heavy Flavours}
\subsection{$e^+e^-$ Collisions}
At low energy, the main process is an annihilation into a virtual photon $\gamma^*$. 
\begin{center}
\begin{tikzpicture}
  \begin{feynman}
    \vertex (a);
    \vertex [above left=of a] (b) {$e^-$};
    \vertex [below left=of a] (c) {$e^+$};
    \vertex [right=of a] (d) {$\gamma^*$};
    % \vertex [below left=1cm and 0.4cm of c] (d);
    % \vertex [above left=1cm and 0.4cm of b] (a);
    % \vertex [left=of a] (i1) {$n$};
    % \vertex [left=of d] (i2) {$n$};
    % \vertex[above right of=a] (f1) {$p$};
    % \vertex[right of=b] (f2) {$e^-$};
    % \vertex[right of=c] (f3) {$e^-$};
    % \vertex[below right of=d] (f4) {$p$};

    \diagram* {
    (b) -- [fermion] (a),
    (a) -- [fermion] (c),
    (a) -- [photon] (d),
      % (a) -- [photon, edge label=$W$] (b) -- [majorana, insertion=0.5, edge label=$\nu_M$] (c) -- [photon, edge label=$W$] (d) ,
      % (i1) -- [fermion] (a),
      % (i2) -- [fermion] (d),
      % (a) -- [fermion] (f1),
      % (b) -- [fermion] (f2),
      % (c) -- [fermion] (f3),
      % (d) -- [fermion] (f4),
    };
  \end{feynman}
\end{tikzpicture}
\end{center}
Low energy means that $m_f\ll\sqrt{s}=E_{CM}=2E=m_{\gamma^*}\ll m_Z$, where $m_f$ are the masses of all fermions. The process $e^-e^+\to Z$ resonates at $\sqrt{s}=m_Z$ and becomes dominant.\\
The initial state is neutral, has lepton number equal to zero and spin 1. In the center of mass, the kinematics can be schematized as follows:
\[
e^+=(E,p,0,0) \quad e^-=(E,-p,0,0) \quad \gamma^*=(2E,0,0,0)
\]
At higher energies, it is possible to have $e^-e^+\to\mu^-\mu^+$. Still in the center of mass, we have:
\[
e^+=(E,p,0,0) \quad e^-=(E,-p,0,0) \quad \mu^+=(E,p\cos\theta,p\sin\theta,0) \quad \mu^-=(E,-p\cos\theta,-p\sin\theta,0)
\]
The cross section is given by:
\[
\sigma_{\mu\mu}=\int_{-1}^{+1}d\cos\theta\frac{d\sigma_{\mu\mu}}{d\cos\theta}=\frac{\pi\alpha^2}{2s}\int_{-1}^{+1}d\cos\theta(1+\cos^2\theta)=\frac{4\pi\alpha^2}{3s}
\]
For $0.5\le\sqrt{s}\le50$\,GeV the measurements agree with the predicted $1/s$ behaviour plus some resonances corresponding to $q\overline{q}$. For $\sqrt{s}>50$\;GeV (e.g. LEP) the process is dominated by the $Z$ formation in the s-channel. Define now the quantity $R$:
\[
R=\frac{\sigma(e^+e^-\to\text{hadrons})}{\sigma(e^+e^-\to\mu^+\mu^-)}=3\sum_{\text{quarks}}e_i^2=R(\sqrt{s})
\]
The sum is extended over all the quarks produced at energy $\sqrt{s}$, i.e. $2m_q<\sqrt{s}$.
\[
\begin{aligned}
&\bullet 0<\sqrt{s}<2m_c: &&R=R_{uds}=3\left[\left(\frac{2}{3}\right)^2+\left(-\frac{1}{3}\right)^2+\left(-\frac{1}{3}\right)^2\right]=2\\
&\bullet 2m_c<\sqrt{s}<2m_b: &&R=R_{udsc}=R_{uds}+3\left(\frac{2}{3}\right)^2=\frac{10}{3}\\
&\bullet 2m_b<\sqrt{s}<2m_t: &&R=R_{udscb}=R_{udsc}+3\left(-\frac{1}{3}\right)^2=\frac{11}{3}\\
&\bullet 2m_t<\sqrt{s}<\infty: &&R=R_{udscbt}=R_{udscb}+3\left(\frac{2}{3}\right)^2=5
\end{aligned}
\]
However, reality is more complicated. $q\overline{q}$ resonances are formed at $\sqrt{s}\approx2m_q$ and their decay modes affect the value of $R$. At $\sqrt{s}\approx m_Z$ the \textbf{weak interactions} completely change the scenario.
\subsection{The November Revolution}
The $u,d,s$ quarks have not been predicted. In fact, mesons and baryons have been discovered and later interpreted in terms of their quark content. In November 1974, the groups of Richter (SLAC) and Ting (Brookhaven) simultaneously discovered a new state with mass $\approx3.1$\;GeV and a width much smaller than their respective resolution. The width was measured to be $0.087$\;MeV and this was called $J/\Psi$.\\
The group of Ting measured the inclusive production of $e^+e^-$ pairs in interactions of 30 GeV protons on a plate of beryllium.
\[
p+\text{Be}\to e^+e^-X
\]
The experiment was searching for mass reasonances with $J^P=1^-$  decaying into $e^+e^-$ pairs with the Drell-Yan process.
\begin{center}
\begin{tikzpicture}
  \begin{feynman}
    \vertex (a);
    \vertex [above left=1cm of a] (v1);
    \vertex [left=of v1] (i1);
    \vertex [above=0.125cm of i1] (i2);
    \vertex [above=0.125cm of i2] (i3);
    \vertex [above=0.125cm of i3] (s1);
    \vertex [below left=1cm of a] (v2);
    \vertex [left=of v2] (i4);
    \vertex [below=0.125cm of i4] (i5);
    \vertex [below=0.125cm of i5] (i6);
    \vertex [below=0.125cm of i6] (s2);
    \vertex [right=of a] (b);
    \vertex [above right=0.6cm and 1.5cm of b] (c) {$e^+,\mu^+$};
    \vertex [below right=0.6cm and 1.5cm of b] (d) {$e^-,\mu^-$};
    \vertex [above=0.7cm of b] (f1);
    \vertex [above=0.125cm of f1] (f2);
    \vertex [above=0.125cm of f2] (f3);
    \vertex [above=0.125cm of f3] (sf1);
    \vertex [below=0.7cm of b] (f4);
    \vertex [below=0.125cm of f4] (f5);
    \vertex [below=0.125cm of f5] (f6);
    \vertex [below=0.125cm of f6] (sf2);

    \diagram* {
    (a) -- [photon, edge label=$\gamma^*$] (b),
    (c) -- [fermion] (b),
    (b) -- [fermion] (d),
    (i1) -- [plain] (v1),
    (i4) -- [plain] (v2),
    (v1) -- [fermion] (a),
    (v2) -- [fermion] (a),
    (i2) -- [fermion] (f2),
    (i3) -- [fermion] (f3),
    (i5) -- [fermion] (f5),
    (i6) -- [fermion] (f6),
    (s1) -> [scalar] (sf1),
    (s2) -> [scalar] (sf2),
    };
    \draw [decoration={brace}, decorate] 
    (i1.south west) -- (s1.north west) node [left] {$p$};
    \draw [decoration={brace}, decorate] 
    (s2.south west) -- (i4.north west) node [left] {$N$};
    \draw [decoration={brace}, decorate] 
    (sf1.north east) -- (f2.south east) node [above right] {\scriptsize{spectators}};
    \draw [decoration={brace}, decorate] 
    (f5. north east) -- (sf2.south east) node [below right] {\scriptsize{spectators}};
  \end{feynman}
\end{tikzpicture}    
\end{center}
They used a two arm magnetic spectrometer to measure separately the electrons and the positrons. Leptonic events are rare, so they used very intense beams in order to have an high rejection power to reject hadrons that can fake $e^+e^-$ or $\mu^+\mu^-$.\\ In 1974, up to the highest available energies, $R\approx2$. Instead, at the Cambridge Electron Accelerator measurements found $R\simeq6$. The scanning in energy was performed in steps of 200 MeV and the measured cross section appeared to be constant, not with the expected trend $1/s$. Decreasing the step from 200 MeV to 2.5 MeV increased the resolving power and a resonance appeared. This particle was called $\Psi$.\\
After some discussion, the correct interpretation emerged: the resonance is a bound state of a \textbf{new quark} $c$ and its anti-quark. The \textbf{charm} quark had been proposed in 1970 to exclude flavour changing neutral currents (FCNC). After 1974, many decays have been precisely measured, the present most precise one gives the mass and width:
\[
m(J/\Psi)=3097\;\text{MeV} \quad \Gamma(J/\Psi)=93\;\text{keV}
\]
It was possible to measure many of the $J/\Psi$ quantum numbers. The resonance is asymmetric, therefore there is interference between $J/\Psi$ formation and the usual $\gamma^*$ exchange in the s-channel, hence $J^P=1^-$. The equality BR$(J/\Psi\to\rho^0\pi^0)=$BR$(J/\Psi\to\rho^\pm\pi^\pm)$ implies isospin $I=0$. The $J/\Psi$ decays into an odd number of $\pi$: the G-parity is conserved, therefore $J/\Psi$ decays via strong interaction.\\
The weak neutral current processes between quarks of different flavour (FCNC) are strongly suppressed, e.g. $\Gamma(K_L^0\to\mu^+\mu^-)\ll\Gamma(K^\pm\to\mu^\pm\nu)$. This was explained in 1970 by Glashow, Iliopoulos and Maiani by introducing the charm quark.\\
They predicted a fourth quark identical to the up quark, but with larger mass, $m_c\gg m_u$, carrying a new quantum number $C$, the charm. As for strangeness, $C$ is conserved in strong and EM interactions and violated in weak interactions. The lightest charmed mesons are $c\overline{q}$ or $\overline{c}q$, with $q=uds$, and have a mass of 1500-2000 MeV and $J^P=0^-$. These mesons decay weakly because of their larger mass.\\
Now, let $Q=s,c$ and $q=u,d$. The \textbf{Zweig (OZI) rule} was set empirically before the advent of QCD and it explains why certain decay modes appear less frequently than otherwise might be expected.
\beginbox{Zweig (OZI) Rule}
In the decay of a bound state of heavy quarks $Q$, the final states without $Q$ (\textit{decays with disconnected diagrams}) have suppressed amplitude with respect to \textit{connected decays}. If the only decays allowed are the disconnected ones, the total width is small and the bound state is narrow.
\endbox
An example of such suppressed decays is $\phi(s\overline{s})\to\pi^+\pi^-\pi^0$. It would be expected that this decay mode would dominate over other decays such as $\phi\to K^+K^-$, which have lower Q-values. Instead, the decay into kaons is seen 84\% of the time, suggesting that the decay into pions is suppressed.\\
A $Q\overline{Q}$ state decays preferentially into $(Q\overline{q})(\overline{Q}q)$, for example:
\[
J/\Psi\to D^+D^- \longleftrightarrow (c\overline{c})\to(\overline{d}c)(d\overline{c})
\]
However, this decay is kinematically forbidden since $m_{J/\Psi}<2m_D$. ($c\overline{c}$) then annihilates into gluons: one gluon is forbidden by colour, two gluons is forbidden by C-parity, 3 gluons are allowed. The latter amplitude is proportional to the strong coupling constant $\alpha_S^3$ and its value produces a smaller width for larger masses, i.e. we say that there is the phenomenon of \textbf{running coupling}.\\
An explanation of the OZI rule can be seen from this decrease of the coupling constant with increasing energy. For the OZI suppressed channels, the gluons must have high $q^2$ (at least as much as the rest mass energies of the quarks into which they decay) and so the coupling constant will appear small to these gluons.\\
If the $J/\Psi$ is a bound state $c\overline{c}$ then mesons $c\overline{q}$ and $\overline{c}q$ must exist. In 1976, the Mark I detector started the search for charmed pseudoscalar mesons $D^0$ and $\overline{D}^0$. According to theory, $D$-mesons have a short lifetime therefore the detection strategy was the presence of narrow peaks in the combined mass of the decay products. A first bump at 1865 MeV with a width compatible with the experimental resolution was observed in the combined mass $K^\pm\pi^\pm$, corresponding to $D^0$ and $\overline{D}^0$ decay. Also, the mass $K^\mp\pi^\pm\pi^\pm$ had a bump at 1875 MeV, corresponding to the $D^+$ and $D^-$ decays. Moreover, in agreement with the GIM predictions, no bump was found in $K^\pm\pi^+\pi^-$, which is forbidden.
\subsection{The \nth{3} Family}
Why do we have consecutive families of quarks/leptons differing only in mass? How do they mix? The answer is that nobody knows: the number of families and the mixing matrix are free parameters of the Standard Model.\\
The Standard Model has some non-QCD constraints:
\begin{itemize}
    \item Families must be complete. The existence of a single member, e.g. the $\nu$ or the $l^-$, implies the existence of all the others to avoid anomalies. It requires $\sum_ie_i=0$ where the sum runs on all members $i$ and colors $c$ of the family.
    \item The $Z$ full width constrains the number of light neutrinos.
    \item At least 3 families are necessary to generate a natural mechanism of CP violation.
    \item The number of flavour $n_f$ is free but the number of colours $n_c$ must be 3.
\end{itemize}
The analysis of Mark I data produced another beautiful discovery, the $\tau$ lepton. The selection followed a well-known method, the unbalanced pairs $e^\pm\mu^\mp$. The unbalanced pairs are only used to cross-check the sample. The yield of $e^\pm\mu^\mp$ pairs vs $\sqrt{s}$ immediately points to the threshold $\sqrt{s}=2m_\tau$. This gives $m_\tau\approx1780$\;MeV.\\
Why is it a lepton? At the time, the evidence came from the lack of any other plausible explanation. Today instead, the evidence is solid: the $Z$ and $W$ decay into $(e,\mu,\tau)$ with the same branching ratio, the $\tau$ lifetime and decays have been measured and found in agreement with predictions.\\
The discovery of the $\tau$ started the hunt for the particles of the new family, still unknown: the \textbf{tau neutrino} $\nu_\tau$ and the pair of quarks up and down similar to $u$ and $d$, now called \textbf{top} $t$ and \textbf{bottom} $b$.\\
In 1977 Lederman and collaborators built at Fermilab a spectrometer with two arms, designed to study $\mu^+\mu^-$ pairs produced by interactions of 400 GeV protons on a copper target. This type of events is rare, therefore it requires intense beams and high rejection power against charged hadrons. There is a clear visible excess between 9 and 10 GeV. When the $\mu\mu$ continuum is subtracted, the excess appears as the superimposition of three separate states. These states called $\Upsilon$(1S), $\Upsilon$(2S) and $\Upsilon$(3S) are bound states $b\overline{b}$.\\
The top quark was directly searched in hadron ($Sp\overline{p}S$ at Fermilab) and lepton (Tristan, LEP) colliders, but it was not found until '90s. At the time, the mass limit was $m_t\ge90$\;GeV. At $m_t\approx m_W\pm m_b$ the search changes: the golden discovery channel moves from $W^+\to t\overline{b}\to W^{+*}b\overline{b}$ to $t\to W^+b$.\\
\begin{minipage}{0.5\textwidth}
\begin{center}
\begin{tikzpicture}
  \begin{feynman}
    \vertex (v1);
    \vertex [right=of v1] (v2);
    \vertex [left=of v1] (i1);
    \vertex [below right=of v1] (f1) {$\overline{b}$};
    \vertex [above right= of v2] (f2) {$b$};
    \vertex [below right=of v2] (f3) {$W^{+*}$};

    \diagram* {
    (i1) -- [photon, edge label=$W^+$] (v1),
    (v1) -- [fermion, edge label=$t$] (v2),
    (v1) -- [anti fermion] (f1),
    (v2) -- [fermion] (f2),
    (v2) -- [photon] (f3),
    };
  \end{feynman}
\end{tikzpicture}    
\end{center}
\end{minipage}\hfill
\begin{minipage}{0.5\textwidth}
\begin{center}
\begin{tikzpicture}
  \begin{feynman}
    \vertex (v1);
    \vertex [left=of v1] (i1);
    \vertex [above right= of v1] (f2) {$b$};
    \vertex [below right=of v1] (f3) {$W^+$};

    \diagram* {
    (i1) -- [fermion, edge label=$t$] (v1),
    (v1) -- [fermion] (f2),
    (v1) -- [photon] (f3),
    };
  \end{feynman}
\end{tikzpicture}    
\end{center}
\end{minipage}\\
$m_t$ was first computed from the radiative corrections for $m_W$ and $m_Z$. The LEP data allowed for predictions of $m_t\approx175$\;GeV. At present, we measure $m_t=(173\pm0.4)$\,GeV.\\
In the '90s the search was concluded at Tevatron. The top is produced in pairs via hadronic interactions. In $pp$ and $\overline{p}p$, the probability density function of initial state partons are different: the $q\overline{q}$ decreases from 90\% to 5\%. In the same range, the total cross section increases from 5 to 600 pb. The top quark decays weakly in $W$ and a \textit{down} type quark, i.e. $d,s,b$ with a coupling proportional to the CKM matrix element $V_{tq}$, therefore the most common decay is $t\to bW^+$. Since $\Gamma=\frac{G_F m_t^3}{8\pi\sqrt{2}}\sim2$\;GeV and $\tau_t\sim4\cdot10^{-25}$\;s, the top decays before any hadronic process may happen. In turn, the $W$ decays into all the lepton-neutrino $q\overline{q}$ pairs.\\
% Finally, a simple table with all the quarks and their quantum numbers.
% \begin{table}[h]
%     \centering
%     \begin{tabular}{l|cccccc}
%     \hline
%     \cellcolor{gray!50} & \cellcolor{yellow!50}$u$ & \cellcolor{yellow!50}$d$ & \cellcolor{yellow!50}$c$ & \cellcolor{yellow!50}$s$ & \cellcolor{yellow!50}$t$ & \cellcolor{yellow!50}$b$\\
%     \hline\hline
%     \cellcolor{yellow!50}Baryon number $\pazocal{B}$ & 1/3 & 1/3 & 1/3 & 1/3 & 1/3 & 1/3\\
%     \hline
%     \cellcolor{yellow!50}Electric Charge $Q$ & +2/3 & -1/3 & +2/3 & -1/3 & +2/3 & -1/3\\
%     \hline
%     \cellcolor{yellow!50}Isospin $I$ & 1/2 & 1/2 & 1/2 & 1/2 & 1/2 & 1/2\\
%     \hline
%     \cellcolor{yellow!50}\nth{3} Isospin $I_3$ & +1/2 & -1/2 & +1/2 & -1/2 & +1/2 & -1/2\\
%     \hline
%     \cellcolor{yellow!50}Charm $C$ & 0 & 0 & +1 & 0 & 0 & 0\\
%     \hline
%     \cellcolor{yellow!50}Strangeness $S$ & 0 & 0 & 0 & -1 & 0 & 0\\
%     \hline
%     \cellcolor{yellow!50}Topness $T$ & 0 & 0 & 0 & 0 & +1 & 0\\
%     \hline
%     \cellcolor{yellow!50}Bottomness $B$ & 0 & 0 & 0 & 0 & 0 & -1\\
%     \hline
%     \end{tabular}
%     \caption{Quantum numbers of all known quarks}
%     \label{tabquarks}
% \end{table}
\newpage
\section{Weak Interactions}
In rare occasions, we see \textbf{violations} of those conservation laws valid for strong and EM interactions: these are known as \textbf{weak interactions} (WI), because of their small coupling.\\
WI take place in almost all processes but they have a negligible effect, except in cases otherwise forbidden. Because of them, stable matter contains only $u,d,e^-$, the other quarks and charged leptons are unstable with respect to WI decays. All elementary particles but gluons and photons \textit{see} WI. Most of our knowledge about WI was obtained from decays of particles and neutrino beams.\\
In the SM, those interactions are classified in two types, according to the charge of their carriers.
\beginbox{Types of Currents in the Weak Interactions}
\begin{itemize}
    \item \textbf{Charged currents} (CC), $W^\pm$ exchange. In CC processes, the charge of quarks and leptons changes by $\pm1$ and at the same time there is a variation of flavour, according to the Cabibbo theory.
\begin{center}
\begin{tikzpicture}
  \begin{feynman}
    \vertex (i1) {$\nu_\mu$};
    \vertex [below=of i1] (i2) {$d$};
    \vertex [right=of i1] (v1);
    \vertex [right=of i2] (v2);
    \vertex [right=of v1] (f1) {$\mu^-$};
    \vertex [right=of v2] (f2) {$u/c$};
    \diagram* {
    (i1) -- [fermion] (v1),
    (i2) -- [fermion] (v2),
    (v1) -- [photon, edge label=$W^\pm$] (v2),
    (v1) -- [fermion] (f1),
    (v2) -- [fermion] (f2),
    };
  \end{feynman}
\end{tikzpicture}
\end{center}  
    \item \textbf{Neutral currents} (NC), $Z$ exchange. In the NC case, quark and lepton flavours remain unchanged (no FCNC).
    \begin{center}
\begin{tikzpicture}
  \begin{feynman}
    \vertex (i1) {$\nu_\mu$};
    \vertex [below=of i1] (i2) {$e^\pm$};
    \vertex [right=of i1] (v1);
    \vertex [right=of i2] (v2);
    \vertex [right=of v1] (f1) {$\nu_\mu$};
    \vertex [right=of v2] (f2) {$e^\pm$};
    \diagram* {
    (i1) -- [fermion] (v1),
    (i2) -- [fermion] (v2),
    (v1) -- [photon, edge label=$Z$] (v2),
    (v1) -- [fermion] (f1),
    (v2) -- [fermion] (f2),
    };
  \end{feynman}
\end{tikzpicture}
\end{center} 
\end{itemize}
\endbox
In the '60s, Glashow, Salam, Weinberg and others developed a theory that unifies WI and EM, which is today part of the SM. The SM was conceived before the discovery of NC, so the existence of NC and its carrier, predicted by the SM in the '60s and directly observed at CERN in 1983, were among the first great success of the SM.
\subsection{Charged Currents}
The modern theory of CC interactions is a successor of the Fermi theory of $\beta$-decay, which describes a point-like interaction proportional to the coupling $G_F$.\\
However, this theory has some intrinsic problems: it is not renormalizable, which means that the cross section violates unitarity at high energy. The SM expands the point-like interaction, introducing a \textbf{heavy charged mediator} $W^\pm$. The SM is mathematically consistent and, more important, reproduces the experimental data with unprecedented accuracy.\\
\begin{minipage}{0.5\textwidth}
\begin{center}
\begin{tikzpicture}
  \begin{feynman}
    \vertex [dot] (v) {};
    \vertex [left=of v] (n) {$n$};
    \vertex [above right=of v] (p) {$p$};
    \vertex [right=of v] (e) {$e^-$};
    \vertex [below right=of v] (nu) {$\overline{\nu}_e$};
    \vertex [above left=0.75em and 0.5em of v] {$G_F$};
    \diagram* {
    (n) -- [fermion] (v),
    (v) -- [fermion] (p),
    (v) -- [fermion] (e),
    (v) -- [fermion] (nu),
    };
  \end{feynman}
\end{tikzpicture}
\end{center}    
\end{minipage}\hfill
\begin{minipage}{0.5\textwidth}
\begin{center}
\begin{tikzpicture}
  \begin{feynman}
    \vertex (v);
    \vertex [left=of v] (d) {$d$};
    \vertex [right=of v] (u) {$u$};
    \vertex [below right=of v] (v1);
    \vertex [right=of v1] (e) {$e^-$};
    \vertex [below right=of v1] (nu) {$\overline{\nu}_e$};

    \diagram* {
    (d) -- [fermion] (v),
    (v) -- [photon, edge label'=$W^-$] (v1),
    (v) -- [fermion] (u)
    (v1) -- [fermion] (e),
    (v1) -- [fermion] (nu),
    };
  \end{feynman}
\end{tikzpicture}
\end{center}    
\end{minipage}
Unlike $\alpha$, $G_F$ is not dimensionless. The EM coupling is proportional to $e^2$, similarly the intensity of the CC $G_F$ is proportional to $g^2$. The matrix elements of the transitions are proportional to the square of of the \textit{weak charge} $g$ and to the propagator:
\[
\pazocal{M}\propto g\frac{1}{Q^2+m_W^2}g\approx\frac{g^2}{m_W^2}\approx G_F\Rightarrow\frac{G_F}{\sqrt{2}}=\frac{g^2}{8m_W^2}
\]
The difference with the EM case is the mass of the carrier: while the photon is massless, the CC carrier is the $W^\pm$ massive particle of spin 1. Therefore, the range of CC turns out to be small $\sim1/m_W$. Moreover, for small values of $Q^2$, the propagator is roughly constant, unlike the case of the massless photon.\\
The most precise value of the Fermi constant $G_F$ is measured by considering the muon decay $\mu^-\to\nu_\mu e^-\overline{\nu}_e$. It is a low energy process approximated by a four-fermion point-like process determined by the Fermi constant. Only leptons are involved and in the approximation that $m_e\approx0$, $m_\mu$ is the only scale of the decay. By dimensional analysis, we get:
\[
\Gamma(\mu^-\to e^-\overline{\nu}_e\nu_\mu)=\frac{1}{\tau_\mu}\propto G_F^2m_\mu^5 \quad \text{Correct computations:}\,\Gamma(\mu^-\to e^-\overline{\nu}_e\nu_\mu)=\frac{G_F^2m_\mu^5}{192\pi^3}(1+\varepsilon)
\]
where $\varepsilon$ is small and depends on radiative corrections and the electron mass, neglected in this case. The mass of the muon and its lifetime are measured with great precision, giving us:
\beginbox{Fermi Constant}
\[
G_F=(1.16637\pm0.00001)\cdot10^{-5}\;\text{GeV$^{-2}$}
\]
\endbox
\subsection{Lepton Universality}
Is the weak CC the same for all leptons and quarks? Do they share the same coupling constant for all the processes? This is true for leptons, but it needs some refinement for quarks.\\
The $e-\mu$ universality is measured by analysing the leptonic decays of of the $\tau^\pm$:
\[
\Gamma(\tau^-\to l^-\overline{\nu}_l\nu_\tau):=\Gamma_l^\tau=\frac{g_\tau^2g_l^2}{m_W^2m_W^2}m_\tau^5\rho_l \quad \text{BR}(\tau^-\to l^-\overline{\nu}_l\nu_\tau):=\text{BR}^\tau_l=\frac{\Gamma_l^\tau}{\Gamma^\tau_{\text{tot}}}
\]
$l$ denotes the lepton, either $e^-$ or $\mu$, $\rho_l$ is the phase space factor. It follows that:
\[
\frac{\Gamma_\mu^\tau}{\Gamma_e^\tau}=\frac{\text{BR}_\mu^\tau}{\text{BR}_e^\tau}=\frac{g_\mu^2\rho_\mu}{g_e^2\rho_e}\to\frac{\text{BR}_\mu^\tau}{\text{BR}_e^\tau}\Bigr|_{\substack{\text{measured}}}=\frac{(17.36\pm0.05)\%}{(17.84\pm0.05)\%}=0.974\pm0.004
\]
Taking into account the values of $\rho_\mu$ and $\rho_e$, we obtain:
\beginbox{Electron-Muon Universality}
\[
\frac{g_\mu}{g_e}\Bigr|_{\substack{\text{measured}}}=1.001\pm0.002
\]
\endbox
The measurement of the $\mu-\tau$ universality is similar. The branching ratio $\mu^-\to e^-\overline{\nu}_e\nu_\mu$ is approximately 100\%, so we can compute:
\[
\frac{\Gamma(\mu^-\to e^-\overline{\nu}_e\nu_\mu)}{\Gamma(\tau^-\to e^-\overline{\nu}_e\nu_\tau)}=\frac{g_\mu^2m_\mu^5\rho_\mu}{g_\tau^2m_\tau^5\rho_\tau}=\frac{\tau_\tau}{\tau_\mu}\frac{\text{BR}(\mu^-\to e^-\overline{\nu}_e\nu_\mu)}{\text{BR}(\tau^-\to e^-\overline{\nu}_e\nu_\tau)}
\]
From this, it follows that:
\[
\frac{g_\mu^2}{g_\tau^2}=\frac{\tau_\tau}{\tau_\mu}\frac{1}{\text{BR}(\tau^-\to e^-\overline{\nu}_e\nu_\tau)}\frac{m_\tau^5\rho_\tau}{m_\mu^5\rho_\mu}
\]
Combining this computation with measurements, we finally get:
\beginbox{Muon-Tau Universality}
\[
\frac{g_\mu}{g_\tau}\Bigr|_{\substack{\text{measured}}}=1.001\pm0.003
\]
\endbox
When we discussed the third family, we said that the $\tau^\pm$ was most likely a lepton: this is a strong confirmation of it.\\
A more ambitious test was to extend the universality to $\tau$ hadronic decays. Consider the following decay modes:
\[
\tau^-\to e^-\overline{\nu}_e\nu_\tau \quad \tau^-\to\mu^-\overline{\nu}_\mu\nu_\tau \quad \tau^-\to\overline{u}d\nu_\tau
\]
From the branching ratios, one expects:
\[
\Gamma_{\tau\to e}\approx\Gamma_{\tau\to\mu}\approx\frac{1}{3}\Gamma_{\tau\to\overline{u}d}
\]
in excellent agreement with universality and with the presence of the colour in the hadronic sector (that's why there is the factor 1/3). A similar test on lepton universality has been performed at LEP in the decays of the $Z$, a NC process. The experiments have measured the decay of the $Z$ into fermion-anti-fermion pairs, finding:
\[
\begin{aligned}
Z\to e^-e^+&:\qquad\mu^-\mu^+&&:\quad\tau^-\tau^+\\
1.000&:\;\;\;1.000\pm0.004&&:\;0.999\pm0.005
\end{aligned}
\]
The total amount of information is impressive and essentially no margin is left to any alternative theory.
\subsection{Parity Violation}
The effect was proposed in 1956 and immediately verified by Madame Wu experiment. The historical reason was a review of weak interaction processes and the explanation of the so-called $\theta-\tau$ puzzle, in modern terms the $K^0$ decay, $(K^0\to2\pi)$ vs $(K^0\to3\pi)$.\\
It was found that \textbf{parity conservation} in weak decays was \textbf{not} really supported by measurements. The CC is V-A which is an acronym for the factor $\gamma_\mu(1-\gamma_5)$ in the current, i.e. the CC have a \textit{preference} for \textbf{left-handed particles} and \textbf{right-handed anti-particles}: these effects clearly violates parity.\\
Consider for example a neutrino $\nu$ with helicity $h=-1$. The parity operator turns it into a $\nu$ with $h=+1$ which does not exist. The conservation is restored when applying also charge conjugation, anti-neutrinos with $h=+1$ do exist. However, this discussion holds only if neutrinos are massless or in the ultra-relativistic approximation where $m_\nu\ll E_\nu$.\\
In 1958, Goldhaber, Grodzins and Sunyar measured the helicity of $\nu_e$, which gave a crucial confirmation of the V-A theory. They used metastable Europium $^{152}$Eu which decays via electronic capture into excited Samarium Sm* emitting an electron neutrino $\nu_e$.
\[
^{152}\text{Eu}+e^-\to^{152}\text{Sm*}+\nu_e
\]
The excited Samarium Sm* decays again into more stable Sm, emitting a photon $\gamma_1$.
\[
^{152}\text{Sm*}\to^{152}\text{Sm}+\gamma_1
\]
The photons emitted in this way are then used to excite another Sm, producing $\gamma_2$ which are finally detected. 
\[
\gamma_1+^{152}\text{Sm}\to^{152}\text{Sm*}\to^{152}\text{Sm}+\gamma_2
\]
The helicity of the neutrino can be determined by looking at the spin of the elements involved in the process. $^{152}$Eu has spin 0, the neutrino has spin 1/2 and $^{152}$Sm* has spin-parity $1^-$. It follows that $^{152}$Sm* has opposite spin with respect to the neutrino, hence they have the same helicity.\\
In the following gamma emission, the photon has spin-parity $1^-$ and $^{152}$Sm is an even-even nucleus, in the state $0^+$. In this case, the helicity of the photon corresponds to the helicity of $^{152}$Sm*, hence to the helicity of the neutrino. By measuring the helicity of such photons, it is possible to measure the helicity of the neutrino.\\
The final result gives $h(\nu_e)=-1.0\pm0.3$, consistent with V-A only.
\subsection{Weak Decays}
The $\pi^\pm$ is the lightest hadron, therefore it may only decay through semi-leptonic CC weak processes:
\[
\pi^+\to\mu^+\nu_\mu \quad \pi^+\to e^+\nu_e
\]
However, in reality it decays only into muons, the electron decay is suppressed roughly by a factor 8000. The reason of this suppression is the \textbf{helicity}: in the $\pi^+$ reference frame, the momenta of $l^+$ and $\nu_l$ must be opposite. Since $\pi^+$ has spin 0, also the spins of $l^+$ and $\nu_l$ must be opposite, therefore the two particles must have the same helicity. The neutrino must have a negative helicity, so the lepton is also forced to have a negative helicity: the transition is suppressed by a factor $(1-\beta_l)$. The $e^+$ is ultra-relativistic while the $\mu^+$ has a smaller $\beta$ due to its larger mass.
% \subsection{$\beta$-Decay}
% For point-like fermions, CC is V-A both for leptons and quarks. However, nucleons and hyperons $(p, n, \Lambda, \Sigma, \Xi, \Omega)$ are bound states of non-free quarks. In this case, strong interaction corrections are important and modify the V-A dynamics.\\
% In the Fermi theory, CC are classified according to the properties of the transition operator. In neutron $\beta$-decay, the electron-neutrino pair may be created as a spin singlet $(S=0)$ or triplet $(S=1)$. In the case of no orbital angular momentum, there are two possibilities to conserve the total angular momentum:
% \begin{itemize}
%     \item \textbf{Fermi transitions}: $S=0, \Delta J_{e\nu}=0$. The direction of the spin of the nucleus remains unchanged, i.e. the interaction takes place with vector coupling $G_V$.
%     \item \textbf{Gamow-Teller transitions}: $S=1, \Delta J_{e\nu}=0,\pm1$. The direction of the spin of the nucleon is flipped, i.e. the transition happens with axial-vector coupling $G_A$
% \end{itemize}
% In principle, these two processes are different, there is no reason why the coupling should be similar or even related.\\
% Assume now that $p,n, e^\pm$ and $\nu$ are spin-1/2 fermions but only neutrinos with $h=-1$ exist. The most general matrix element for the four-body interaction is:
% \[
% \pazocal{M}=\frac{G_F}{\sqrt{2}}\sum_j C_j[\overline{u}_pO_ju_n]\left[\overline{u}_eO_j\left(\frac{1-\gamma_5}{2}\right)u_\nu\right]
% \]
% where $O_j$ are the current operators with given properties: S for scalar, P for pseudo-scalar, V for vector, A for axial-vector, T for tensor. The pseudo-scalar case is irrelevant, since it can only be built from the proton velocity in the neutron rest frame and it is suppressed. From comparison with data, we can exclude the S and V terms, due to lack of interference between them they cannot be both present. Same argument holds for A and T. The angular distribution of the electrons are consistent only with V for Fermi transitions and with A for Gamow-Teller transitions. The matrix element now becomes:
% \[
% \pazocal{M}=\frac{G_F}{\sqrt{2}}[\overline{u}_p\gamma^\mu(C_V+C_A\gamma_5)u_n]\left[\overline{u}_e\gamma^\mu\left(\frac{1-\gamma_5}{2}\right)u_\nu\right]
% \]
% The value of $C_V$ can be measured by comparing composite hadrons with free pure V-A leptons and it turns out to be $C_V\approx1$. $(C_A)^2$ can be measured from the relative strength of Fermi and Gamow-Teller transitions, by comparing the neutron $\beta$-decay with a pure Fermi, $^{14}$O$\to^{14}$N$e^+\nu$, obtaining $|C_A|\simeq1.267$. The sign can be measured from the polarization of protons and it turns out to be negative.
\subsection{Quark Decays}
For high mass quarks and at high $Q^2$, the structure V-A seems restored and quarks behave as free point-like particles. However, with more accurate data some discrepancies appear, not due to strong interactions. Some tiny but well measured effects seem to contradict the universality: $G_F$ turns out to be slightly larger for leptons than for quarks.
\[
G_F(\mu^-\to e^-\overline{\nu}_e\nu_\mu)\approx1.166\cdot10^{-5}\,\text{GeV$^{-2}$} \quad G_F(d\to ue^-\overline{\nu}_e)\approx1.136\cdot10^{-5}\,\text{GeV$^{-2}$}
\]
In 1963, Cabibbo came up with a theory to explain this effect, introducing the so-called \textbf{Cabibbo angle} $\theta_c$. The idea was the following: hadrons are built up with quarks $u,d,s$ ($c,b,t$ had not been discovered at the time). However, in the CC processes, the quarks $(ds)$ \textbf{mix together}, i.e. rotate by an angle $\theta_c$, in such a way the the CC process sees rotated quarks $(d's')$.
\beginbox{Cabibbo Theory}
\[
\begin{pmatrix}
    d'\\s'
\end{pmatrix}=\begin{pmatrix}
    \cos\theta_c & \sin\theta_c\\
    -\sin\theta_c & \cos\theta_c
\end{pmatrix}\begin{pmatrix}
    d\\s
\end{pmatrix}
\]
\endbox
Therefore, with respect to the strength of the leptonic process, the $ud$ coupling is reduced by a factor $\cos\theta_c$ and the $us$ coupling by a factor $\sin\theta_c$.\\
\begin{minipage}{0.3\textwidth}
 \begin{tikzpicture}
  \begin{feynman}
    \vertex (a);
    \vertex [above left=of a] (b) {$l^-$};
    \vertex [below left=of a] (c) {$\overline{\nu}_l$};
    \vertex [right=of a] (d) {$\propto 1$};
    % \vertex [below left=1cm and 0.4cm of c] (d);
    % \vertex [above left=1cm and 0.4cm of b] (a);
    % \vertex [left=of a] (i1) {$n$};
    % \vertex [left=of d] (i2) {$n$};
    % \vertex[above right of=a] (f1) {$p$};
    % \vertex[right of=b] (f2) {$e^-$};
    % \vertex[right of=c] (f3) {$e^-$};
    % \vertex[below right of=d] (f4) {$p$};

    \diagram* {
    (b) -- [fermion] (a),
    (c) -- [anti fermion] (a),
    (a) -- [photon, edge label'=$W^\pm$] (d),
      % (a) -- [photon, edge label=$W$] (b) -- [majorana, insertion=0.5, edge label=$\nu_M$] (c) -- [photon, edge label=$W$] (d) ,
      % (i1) -- [fermion] (a),
      % (i2) -- [fermion] (d),
      % (a) -- [fermion] (f1),
      % (b) -- [fermion] (f2),
      % (c) -- [fermion] (f3),
      % (d) -- [fermion] (f4),
    };
  \end{feynman}
\end{tikzpicture}
\end{minipage}\hfill
\begin{minipage}{0.3\textwidth}
 \begin{tikzpicture}
  \begin{feynman}
    \vertex (a);
    \vertex [above left=of a] (b) {$u$};
    \vertex [below left=of a] (c) {$d$};
    \vertex [right=of a] (d) {$\propto \cos\theta_c$};
    % \vertex [below left=1cm and 0.4cm of c] (d);
    % \vertex [above left=1cm and 0.4cm of b] (a);
    % \vertex [left=of a] (i1) {$n$};
    % \vertex [left=of d] (i2) {$n$};
    % \vertex[above right of=a] (f1) {$p$};
    % \vertex[right of=b] (f2) {$e^-$};
    % \vertex[right of=c] (f3) {$e^-$};
    % \vertex[below right of=d] (f4) {$p$};

    \diagram* {
    (b) -- [fermion] (a),
    (c) -- [anti fermion] (a),
    (a) -- [photon, edge label'=$W^\pm$] (d),
      % (a) -- [photon, edge label=$W$] (b) -- [majorana, insertion=0.5, edge label=$\nu_M$] (c) -- [photon, edge label=$W$] (d) ,
      % (i1) -- [fermion] (a),
      % (i2) -- [fermion] (d),
      % (a) -- [fermion] (f1),
      % (b) -- [fermion] (f2),
      % (c) -- [fermion] (f3),
      % (d) -- [fermion] (f4),
    };
  \end{feynman}
\end{tikzpicture}
\end{minipage}\hfill
\begin{minipage}{0.3\textwidth}
 \begin{tikzpicture}
  \begin{feynman}
    \vertex (a);
    \vertex [above left=of a] (b) {$u$};
    \vertex [below left=of a] (c) {$s$};
    \vertex [right=of a] (d) {$\propto \sin\theta_c$};
    % \vertex [below left=1cm and 0.4cm of c] (d);
    % \vertex [above left=1cm and 0.4cm of b] (a);
    % \vertex [left=of a] (i1) {$n$};
    % \vertex [left=of d] (i2) {$n$};
    % \vertex[above right of=a] (f1) {$p$};
    % \vertex[right of=b] (f2) {$e^-$};
    % \vertex[right of=c] (f3) {$e^-$};
    % \vertex[below right of=d] (f4) {$p$};

    \diagram* {
    (b) -- [fermion] (a),
    (c) -- [anti fermion] (a),
    (a) -- [photon, edge label'=$W^\pm$] (d),
      % (a) -- [photon, edge label=$W$] (b) -- [majorana, insertion=0.5, edge label=$\nu_M$] (c) -- [photon, edge label=$W$] (d) ,
      % (i1) -- [fermion] (a),
      % (i2) -- [fermion] (d),
      % (a) -- [fermion] (f1),
      % (b) -- [fermion] (f2),
      % (c) -- [fermion] (f3),
      % (d) -- [fermion] (f4),
    };
  \end{feynman}
\end{tikzpicture}
\end{minipage}\\
Processes with $\Delta S=0$ are proportional to $\cos^2\theta_c$ and those with $\Delta S=1$ to $\sin^2\theta_c$. Even processes proportional to $\sin^4\theta_c$ may happen, but we say that they are Cabibbo doubly suppressed. All the anomalies come back under control if $\sin^2\theta_c\approx0.03$ and $\cos^2\theta_c\approx0.97$.\\
In this context, the GIM mechanism was proposed to explain the absence of flavour changing neutral currents (FCNC). At the time, they knew that:
\[
\begin{aligned}
&\text{BR}(K^0\to\mu^+\mu^-)=7\cdot10^{-9} &&\text{BR}(K^+\to\mu^+\nu_\mu)=0.64\\
&\text{BR}(K^+\to\pi^+\nu\overline{\nu})=(1.5^{+1.3}_{-0.9})\cdot10^{-10} &&\text{BR}(K^+\to\pi^0e^+\nu_e)=(4.98\pm0.07)\cdot10^{-2}
\end{aligned}
\]
which means a factor $\sim10^{-8}$ between NC and CC decays. If the $Z$, the carrier of NC, sees the same quark mixture as the $W^\pm$ in the CC, then the CC decay would be suppressed only by a factor 5\%. The idea was to introduce a \textbf{fourth quark}, the charm $c$ with charge +2/3 to solve this problem.\\
In the GIM mechanism, NC contains four hadronic terms coupled with the $Z$.\\
\begin{center}
    \begin{tikzpicture}
    \begin{feynman}
    \vertex (a);
    \vertex [left=of a] (b);
    \vertex [above right=of a] (c) {$q=u,d',c,s'$};
    \vertex [below right=of a] (d) {$\overline{q}=\overline{u},\overline{d}',\overline{c},\overline{s}'$};
    \diagram*{
    (a) -- [photon, edge label'=$Z$] (b),
    (a) -- [anti fermion] (c),
    (a) -- [fermion] (d),
    };
    \end{feynman}
    \end{tikzpicture}
\end{center}
Assuming the validity of Cabibbo theory and summing all terms we get:
\begin{align*}
u\overline{u}+d'\overline{d}'+c\overline{c}+s'\overline{s}'=&u\overline{u}+(d\cos\theta_c+s\sin\theta_c)(\overline{d}\cos\theta_c+\overline{s}\sin\theta_c)\\
&+c\overline{c}+(s\cos\theta_c-d\sin\theta_c)(\overline{s}\cos\theta_c-\overline{d}\sin\theta_c)\\
=&u\overline{u}+d\overline{d}+c\overline{c}+s\overline{s}
\end{align*}
The non-diagonal terms, which induce FCNC, disappear.\\
Why is $\text{BR}(K^0\to\mu^+\mu^-)$ small but not zero?\\ Technically, it is a \nth{2} order CC, proportional to $g^4\sin\theta_c\cos\theta_c$ (diagram on the left). The final state is the same as a \nth{1} order FCNC but it is incompatible with data. This is cured by introducing the $c$ quark (diagram on the right), whose contribution cancels the ones coming from FCNC in the limit $m_c\to m_u$. The data on this process put limits on $m_c$ between 1 and 3 GeV.\\
\begin{minipage}{0.5\textwidth}
\begin{center}
    \begin{tikzpicture}
    \begin{feynman}
    \vertex (v1);
    \vertex [below=of v1] (v2);
    \vertex [left=of v1] (i1) {$d$};
    \vertex [left=of v2] (i2) {$\overline{s}$};
    \vertex [right=of v1] (v3);
    \vertex [right=of v2] (v4);
    \vertex [right=of v3] (f1) {$\mu^-$};
    \vertex [right=of v4] (f2) {$\mu^+$};
    \vertex [above=0.05cm of v1] () {$\cos\theta_c$};
    \vertex [below=0.05cm of v2] () {$\sin\theta_c$};
    \diagram*{
    (i1) -- [fermion] (v1),
    (i2) -- [anti fermion] (v2),
    (v1) -- [fermion, edge label'=$u$] (v2),
    (v1) -- [photon, edge label'=$W^-$] (v3),
    (v2) -- [photon, edge label=$W^+$] (v4),
    (v4) -- [fermion] (v3),
    (v3) -- [fermion] (f1),
    (v4) -- [anti fermion] (f2),
    };
    \end{feynman}
    \end{tikzpicture}
\end{center}
\end{minipage}\hfill
\begin{minipage}{0.5\textwidth}
\begin{center}
    \begin{tikzpicture}
    \begin{feynman}
    \vertex (v1);
    \vertex [below=of v1] (v2);
    \vertex [left=of v1] (i1) {$d$};
    \vertex [left=of v2] (i2) {$\overline{s}$};
    \vertex [right=of v1] (v3);
    \vertex [right=of v2] (v4);
    \vertex [right=of v3] (f1) {$\mu^-$};
    \vertex [right=of v4] (f2) {$\mu^+$};
    \vertex [above=0.05cm of v1] () {$-\sin\theta_c$};
    \vertex [below=0.05cm of v2] () {$\cos\theta_c$};
    \diagram*{
    (i1) -- [fermion] (v1),
    (i2) -- [anti fermion] (v2),
    (v1) -- [fermion, edge label'=$c$] (v2),
    (v1) -- [photon, edge label'=$W^-$] (v3),
    (v2) -- [photon, edge label=$W^+$] (v4),
    (v4) -- [fermion] (v3),
    (v3) -- [fermion] (f1),
    (v4) -- [anti fermion] (f2),
    };
    \end{feynman}
    \end{tikzpicture}
\end{center}
\end{minipage}
In 1973, Kobayashi and Maskawa extended the Cabibbo scheme to the new generation of quarks. The new mixing matrix is a \textbf{three-dimension unitary matrix} with three real parameters and one imaginary phase:
\beginbox{CKM Matrix}
\[
\begin{pmatrix}
    d'\\s'\\b'
\end{pmatrix}=\begin{pmatrix}
    V_{ud} & V_{us} & V_{ub}\\
    V_{cd} & V_{cs} & V_{cb}\\
    V_{td} & V_{ts} & V_{tb}\\
\end{pmatrix}\begin{pmatrix}
    d\\s\\b
\end{pmatrix}
\]
\endbox
This is known as the \textbf{CKM matrix}. They observed that the CP violation, already discovered, is automatically generated by the matrix when the phase is non-zero. Moreover, the nine elements of this matrix govern the flavour changes in CC processes.
\newpage
\section{$K^0$ Mesons and CKM Matrix}
\subsection{$K^0$ Processes}
The neutral mesons $K^0$ and $\overline{K}^0$ are special quark systems in which unusual phenomena take place. They are produced by strong interactions with fixed strangeness:
\[
\ket{K^0}=\ket{d\overline{s}}, S=+1 \quad \ket{\overline{K}^0}=\ket{s\overline{d}}, S=-1
\]
Simple kinematics shows that it is possible to produce a pure sample of $K^0$. Consider the reaction $ab\to cd$, if $(m_c+m_d)>(m_a+m_b)$ it requires some kinetic energy to happen. Study the process in the system where the target $b$ is at rest:
\[
s\Bigr|_{\substack{\text{LAB}}}^{\text{ini}}=m_a^2+m_b^2+2E_a^{\min} m_b=s\Bigr|_{\substack{\text{CM}}}^{\text{fin}}=(m_c+m_d)^2\Rightarrow E_a^{\min}=\frac{(m_c+m_d)^2-(m_a^2+m_b^2)}{2m_b}
\]
The process $\pi^- p\to\Lambda K^0$ has a threshold energy $E_{\pi^-}^{\min}=0.91$\;GeV, to compare it with\\
$\pi^- p\to K^0\overline{K}^0n$ which has $E_{\pi^-}^{\min}=1.50$\;GeV. It follows that for $0.91<E_{\pi^-}<1.50$\;GeV only $K^0$ are produced.\\
However, even when selecting pure $K^0$ some unexpected $\overline{K}^0$ show up in subsequent processes. This demonstrates that production and decay of these mesons follow complicated rules. The following strong interactions are allowed because they conserve strangeness:
\[
(1)\,K^+n\to K^0p \quad (2)\,K^-p\to\overline{K}^0n \quad (3)\,K^0p\to K^+n \quad (4)\,\overline{K}^0p\to\pi^0\Sigma^+
\]
Sometimes, the $K^0$ generated in (1) re-interacts as a $\overline{K}^0$ via reaction (4) or the same with (2) and (3). It seems that there are transitions, i.e. \textbf{oscillations}, in flight $K^0\leftrightarrow\overline{K}^0$. Notice that transitions $n\leftrightarrow\overline{n}$ are forbidden because of baryon number and $e^+\leftrightarrow e^-$ are forbidden because of electric charge and lepton number. These quantum numbers are conserved in all interactions, instead the $K^0\leftrightarrow\overline{K}^0$ oscillations are only forbidden by strangeness conservation, which is conserved in strong interactions but \textbf{violated} in weak interactions.
\subsection{$K^0$ and $\overline{K}^0$ Decays}
Moreover, the decay of $K^0$ and $\overline{K}^0$ was not understood and created a puzzle. Both $K^0$ and $\overline{K}^0$ can decay into $\pi^+\pi^-$ and $\pi^+\pi^-\pi^0$, two states with different G-parity but we know that G-parity is not conserved in WI. The explanation was provided by Gell-Mann and Pais before the discovery that WI violates parity.\\
$K^0$ and $\overline{K}^0$ are eigenstates of the strong interaction, each one is the anti-particle of the other. They have opposite strangeness but WI does not conserve strangeness and see a mixture of $K^0$ and $\overline{K}^0$. This mixture is interpreted as two new states, a \textbf{superposition} of $K^0$ and $\overline{K}^0$. These two new states are not a particle-anti-particle pair so they may have in principle different properties. If the mass difference allows that, they might oscillate one into the other.\\
The two $K^0$ states have different values of CP, respectively CP=+1 and CP=$-$1. The first decays into $2\pi$ and the second into $3\pi$. The $3\pi$ state is near the kinematic threshold\\
($m_K\approx3m_\pi+70$\;MeV) so the lifetime of this state is much longer than the lifetime of the $2\pi$ state: the proposal was to call \textbf{short} the CP=+1 state and \textbf{long} the CP=$-$1 state.
\[
K_S^0: \text{CP}=+1, \tau=0.895\cdot10^{-10}\;\text{s} \quad K_L^0: \text{CP}=-1, \tau=0.512\cdot10^{-7}\;\text{s}
\]
The $K_L^0$ was first observed in 1956 by Lande and collaboration with a cloud chamber. The path between the 3 GeV proton beam and the cloud chamber was 6 meters, long enough for the decay of all strange particles known at the time. They observed interactions with nuclei of He, producing a final state with $S\neq0$, like $\overline{K}^0 \text{He}\to\Sigma^-ppn\pi^+$. These final states cannot be generated by a $K^0$ because of the value of $S$ but no $\overline{K}^0$ should be present, because the primary proton energy was chosen to be below the energy threshold for $\overline{K}^0$ production: for some reasons, $\overline{K}^0$ mesons have appeared.\\
Both $K^0$ and $\overline{K}^0$ are strong interaction and parity eigenstates but not C or CP eigenstates. 
\[
\begin{aligned}
&C\ket{K^0}=-\ket{\overline{K}^0} &&CP\ket{K^0}=+\ket{\overline{K}^0}\\    
&C\ket{\overline{K}^0}=-\ket{K^0} &&CP\ket{\overline{K}^0}=+\ket{K^0}
\end{aligned}
\]
We define $\ket{K_1^0}$ and $\ket{K_2^0}$ as linear combinations of $K^0$ and $\overline{K}^0$:
\[
\left\{
\begin{aligned}
&\ket{K_1^0}=\frac{1}{\sqrt{2}}\left[\ket{K^0}+\ket{\overline{K}^0}\right]\\    
&\ket{K_2^0}=\frac{1}{\sqrt{2}}\left[\ket{K^0}-\ket{\overline{K}^0}\right]
\end{aligned}
\right.
\longleftrightarrow
\left\{
\begin{aligned}
&\ket{K^0}=\frac{1}{\sqrt{2}}\left[\ket{K_1^0}+\ket{K_2^0}\right]\\ 
&\ket{\overline{K}^0}=\frac{1}{\sqrt{2}}\left[\ket{K_1^0}-\ket{K_2^0}\right]
\end{aligned}
\right.
\]
These are CP eigenstates, CP$\ket{K_1^0}=+\ket{K_1^0}$ and CP$\ket{K_2^0}=-\ket{K_2^0}$. Look now at the $2\pi$ and $3\pi$ states.
\beginbox{CP Eigenstates}
\begin{multicols}{2}
\begin{itemize}
    \item Consider the $\ket{\pi^0\pi^0}$ state:
    \begin{itemize}
        \item P$\ket{\pi^0\pi^0}=+\ket{\pi^0\pi^0}$
        \item C$\ket{\pi^0\pi^0}=+\ket{\pi^0\pi^0}$
    \end{itemize}
    Hence, $\ket{\pi^0\pi^0}$ has CP=+1
    \item If $L=S_1=S_2=0$, $\pi^+\pi^-$ also have parity +1 and therefore CP=+1
    \item We can conclude that CP$(2\pi)=+1$ both for $\ket{\pi^0\pi^0}$ and $\ket{\pi^+\pi^-}$.
\end{itemize}
\columnbreak
\begin{itemize}
    \item Consider the $3\pi$ state $\ket{\pi^0\pi^0\pi^0}$:
    \begin{itemize}
        \item P$\ket{\pi^0\pi^0\pi^0}=-\ket{\pi^0\pi^0\pi^0}$
        \item C$\ket{\pi^0\pi^0\pi^0}=+\ket{\pi^0\pi^0\pi^0}$
    \end{itemize}
    Hence, $\ket{\pi^0\pi^0\pi^0}$ has CP=$-$1
    \item For the $\ket{\pi^+\pi^-\pi^0}$ state now:
    \begin{itemize}
        \item P$\ket{\pi^+\pi^-\pi^0}=-\ket{\pi^+\pi^-\pi^0}$
        \item C$\ket{\pi^+\pi^-\pi^0}=+\ket{\pi^+\pi^-\pi^0}$
    \end{itemize}
    Therefore $\ket{\pi^+\pi^-\pi^0}$ has CP=-1
    \item CP($3\pi)=-1$ both for $\ket{\pi^0\pi^0\pi^0}$ and $\ket{\pi^+\pi^-\pi^0}$.
\end{itemize}
\end{multicols}
\endbox
Therefore, we identify $K_1^0$ with $K_S^0$ and $K_2^0$ with $K_L^0$.
\subsection{$K^0$ Oscillations}
While the $K^0$ and $\overline{K}^0$ masses are equal because of CPT, there is no symmetry regarding the masses of $K_S^0$ and $K_L^0$. The measurement gives:
\[
\Delta m_K=m(K_L^0)-m(K_S^0)=3.510\pm0.018\;\text{$\mu$eV}
\]
The mass difference tells us that the two states evolve with different time constants:
\[
\Psi_S(t)=\Psi_S(0)\exp\left[-\left(\frac{\Gamma_S}{2}+im_S\right)t\right] \quad \Psi_L(t)=\Psi_L(0)\exp\left[-\left(\frac{\Gamma_L}{2}+im_L\right)t\right]
\]
Take now a pure $K^0$ at $t=0$. In the case of no decay, i.e. $\Gamma_{S,L}=0$, the probability to find a $K^0$ or $\overline{K}^0$ is a function of time:
\[
\pazocal{P}_{K^0}(t)=\frac{1}{4}|e^{-im_St}+e^{-im_Lt}|^2=\cos^2\left(\frac{\Delta m_K}{2}t\right) \quad \pazocal{P}_{\overline{K}^0}(t)=\frac{1}{4}|e^{-im_St}-e^{-im_Lt}|^2=\sin^2\left(\frac{\Delta m_K}{2}t\right)
\]
In addition, the oscillations are damped by the occurrence of the decays, dominated by $\Gamma_S$ because of the shorter lifetime.\\
Starting with a pure $K^0$, there is an oscillation between the two states according to $\tau_L, \tau_S$ and $\Delta m_K$. To test these predictions, one should single out $K^0$ and $\overline{K}^0$ in the decay. This cannot be done by using the $2\pi$ or $3\pi$ decays because they have definite CP and not definite strangeness. A possibility is to use semi-leptonic decays which are different for $s$ and $\overline{s}$:
\[
\begin{aligned}
&\overline{s}\to\overline{u}l^+\nu_l\Rightarrow K^0\to\pi^-l^+\nu_l\\
&s\to ul^-\overline{\nu}_l\Rightarrow\overline{K}^0\to\pi^+l^-\overline{\nu}_l
\end{aligned}
\]
The sign of the charged lepton flags the strangeness of the $K^0/\overline{K}^0$. The experimental measure regards the charge asymmetry $\delta$, i.e. the difference between positive charged and negative charged leptons, which is directly correlated to the oscillations. The results are in good agreement with the expectations except for the tail.
\subsection{$K^0$ Regeneration}
The regeneration consists in a clever use of an absorber to demonstrate the superposition of $K^0$ and $\overline{K}^0$. Again, start with a pure $K^0$ beam in vacuum, i.e. equal amounts of $K_S^0$ and $K_L^0$. After $t\approx10\tau_S$, the $K_S^0$ intensity is damped by a factor $e^{-t/\tau_S}\approx45\cdot10^{-6}$. On the other hand, the $K_L^0$ intensity is damped by a factor $e^{-t/\tau_L}\approx0.98$. For $K^0$ with 1 GeV momentum, this corresponds to roughly 0.5m so after this distance we are left with a pure beam of $K_L^0$, 50\% $K^0$ and 50\% $\overline{K}^0$. If we put another target after $t=20\tau_S$, we will get $K^0$ and $\overline{K}^0$ interactions. The two mesons interact differently in the target:
\[
\begin{aligned}
&K^0p\to K^0p/K^+n &&K^0n\to K^0n\\
&\overline{K}^0p\to\overline{K}^0p/\Lambda\pi^+\to\Sigma^0\pi^+/\Sigma^+\pi^0 &&\overline{K}^0n\to\overline{K}^0n/\Lambda\pi^0\to\Sigma^+\pi^-/\Sigma^0\pi^0/\Sigma^-\pi^+
\end{aligned}
\]
The $s$ quark from the $\overline{K}^0$ can swap with one of the quarks in the proton or the neutron but this is not possible for the $\overline{s}$ in the $K^0$. It follows that there are more $\overline{K}^0$ processes and we will no longer have 50\% $K^0+50\%\overline{K}^0$ but an amount of $K_S^0$ is born and we will see $K_S^0$ decays again. 200 $2\pi$ decays were observed, so it was possible to confirm oscillations and regeneration. 
\subsection{CP Violation}
If we observe some decays $K_L^0\to2\pi$ or $K_S^0\to3\pi$ with small but non-zero branching ratio, it would be an experimental evidence of the non-conservation of CP. Since a small amount of $K_S^0\to3\pi$ is not observable due to background, the key observation is $K_L^0\to2\pi$.\\
In 1964, an experiment was built at the Brookhaven AGS, following this scheme: the primary 30 GeV proton beam hits a beryllium target and the secondaries at $\theta=30^\circ$ are selected. If they are charged, they are collimated and bent away while if they are neutral they are collimated and let decay. The resultant $K_L^0$ hit a second lead target, regenerate and let decay again in a long decay tube $\sim18$\;m. No $K_S^0$  are left, so if CP is conserved then we should observe only $3\pi$ decays while if there is a violation $2\pi$ decays should also be present. The final results tell us:
\[
R=\frac{\text{BR}(K_L^0\to\pi^+\pi^-)}{\text{BR}(K_L^0\to\text{charged})}=(2.0\pm0.4)\cdot10^{-3}\Rightarrow\text{CP is violated!}
\]
The $K_L^0\to\pi^+\pi^-$ is not the only decay channel which shows CP violation. Another important process is the semi-leptonic decay $K_L^0\to\pi^\pm l^\mp\nu_l$ with:
\[
\text{BR}(K_L^0\to\pi^\pm e^\mp\nu_e)\approx40.6\% \quad \text{BR}(K_L^0\to\pi^\pm\mu^\mp\nu_\mu)\approx27.0\%
\]
If CP were conserved, the rate with positive and negative charge would be the same since they are connected by a CP transformation. Instead, they are different and it is customary to express the difference as:
\[
\delta_L=\frac{\Gamma(K_L^0\to\pi^- l^+\nu_l)-\Gamma(K_L^0\to\pi^+ l^-\overline{\nu}_l)}{\Gamma(K_L^0\to\pi^- l^+\nu_l)+\Gamma(K_L^0\to\pi^+ l^-\overline{\nu}_l)}
\]
This is measured to be $\delta_L=(3.32\pm0.06)\cdot10^{-3}$.
\subsection{CKM Matrix}
It is possible to reinterpret the CP violation using the CKM matrix previously introduced. In a scheme with 3 families, the matrix requires 3 real rotation angles $\theta_{ij}$ and 1 imaginary phase $\delta$.
\begin{align*}
V_{\text{CKM}}&=\begin{pmatrix}
    V_{ud} & V_{us} & V_{ub}\\
    V_{cd} & V_{cs} & V_{cb}\\
    V_{td} & V_{ts} & V_{tb}\\
\end{pmatrix}=\begin{pmatrix}
    1 & 0 & 0\\
    0 & c_{23} & s_{23}\\
    0 & -s_{23} & c_{23}
\end{pmatrix}
\begin{pmatrix}
    c_{13} & 0 & s_{13}e^{-i\delta}\\
    0 & 1 & 0\\
    -s_{13}e^{+i\delta} & 0 & c_{13}
\end{pmatrix}
\begin{pmatrix}
    c_{12} & s_{12} & 0\\
    -s_{12} & c_{12} & 0\\
    0 & 0 & 1
\end{pmatrix}\\
&=\begin{pmatrix}
    c_{12}c_{13} & s_{12}c_{13} & s_{13}e^{-i\delta}\\
    -s_{12}c_{23}-c_{12}s_{23}s_{13}e^{+i\delta} & c_{12}c_{23}-s_{12}c_{23}s_{13}e^{+i\delta} & s_{23}c_{13}\\
    s_{12}s_{23}-c_{12}c_{23}s_{13}e^{+i\delta} & -_{!2}s_{23}-s_{12}c_{23}s_{13}e^{+i\delta} & c_{23}c_{13}
\end{pmatrix}
\end{align*}
where $c_{ij}:=\cos\theta_{ij}$ and $s_{ij}:=\sin\theta_{ij}$. The violation associated with the CKM matrix are usually studied with the \textbf{Wolfenstein parametrization}, which highlights the small terms and their physical meaning.
\[
V_{\text{CKM}}^{\text{W}}=\begin{pmatrix}
    1-\frac{\lambda^2}{2} & \lambda & A\lambda^3(\rho-i\eta)\\
    -\lambda & 1-\frac{\lambda^2}{2} & A\lambda^2\\
    A\lambda^3(1-\rho-i\eta) & A\lambda^2 & 1
\end{pmatrix}
\]
where $\lambda\simeq s_{12}$, $A\lambda^2\simeq s_{23}$ and $A\lambda^3(\rho+i\eta)\simeq s_{13}e^{+i\delta}$.\\
If $\eta=0$, therefore $\delta=0$, the matrix is \textbf{real} and there is \textbf{no CP violation}. The CP violation in the $K^0$ system can be explained with the CKM formalism. For each of the $K^0\leftrightarrow\overline{K}^0$ diagram, look at the t-channel exchange, with 9 couples of diagrams $(uu,uc,ut,cu,cc,ct,\dots)$. Here we discuss only the $ct$ case.
\[
\pazocal{M}(K^0\to\overline{K}^0)\propto V_{cd}V_{ts}^*V_{cs}^*V_{td} \quad \pazocal{M}(\overline{K}^0\to K^0)\propto V_{cs}^*V_{ts}V_{cs}V_{td}^*
\]
If $V_{ij}$ is real, then $\pazocal{M}(K^0\to\overline{K}^0)=\pazocal{M}(\overline{K}^0\to K^0)$ and there is no CP violation, while if $V_{ij}$ is complex, then $\pazocal{M}(K^0\to\overline{K}^0)\neq\pazocal{M}(\overline{K}^0\to K^0)$ and we have CP violation. In this case, $\pazocal{M}(K^0\to\overline{K}^0)\neq\pazocal{M}(\overline{K}^0\to K^0)$ and their difference is proportional to $i\mathfrak{Im}(V_{td})=iA\lambda^3\eta$.\\
In the SM, CP violation is expected to occur also in the $D^0-\overline{D}^0$ and $B^0-\overline{B}^0$ systems through the same dynamical mechanism. The importance of the phenomenon depends on the value of the CKM matrix, i.e. by the quark mixing: in the first case, the main contribution is from the $b$ quark exchange but the product $V_{cb}V_{ub}$ is very small. In the second case, it is dominated by the $t$ quark exchange and substantial levels of mixing are expected.
\subsection{$\nu$ Oscillations}
Quarks of same charge and different flavours mix together, with the CKM matrix parametrizing this process. What about the lepton sector? Do neutrinos oscillate? The answer is yes. Consider a simple toy model with only two families:
\[
\begin{pmatrix}
    \nu_e\\ \nu_\mu
\end{pmatrix}=\begin{pmatrix}
    \cos\theta_\nu & \sin\theta_\nu\\
    -\sin\theta_\nu & \cos\theta_\nu
\end{pmatrix}\begin{pmatrix}
    \nu_1 \\ \nu_2
\end{pmatrix}
\]
The free parameters are the masses and the mixing angle $\theta_\nu$. The formalism is the same as the one in the $K_1^0\leftrightarrow K_2^0$:
\[
\left\{
\begin{aligned}
&\ket{\nu_e(t)}=\cos\theta_\nu e^{-iE_1t}\ket{\nu_1}+\sin\theta_\nu e^{-iE_2t}\ket{\nu_2}\\
&\ket{\nu_\mu(t)}=-\sin\theta_\nu e^{-iE_1t}\ket{\nu_1}+\cos\theta_\nu e^{-iE_2t}\ket{\nu_2}
\end{aligned}
\right.
\]
The oscillation probability is given by:
\[
\pazocal{P}(\nu_e\to\nu_\mu)=1-|\Braket{\nu_e(t)|\nu_e(0)}|^2
\]
where the matrix element is computed as:
\begin{align*}
|\Braket{\nu_e(t)|\nu_e(0)}|^2=&|(\cos\theta_\nu e^{-iE_1t}\bra{\nu_1}+\sin\theta_\nu e^{-iE_2t}\bra{\nu_2})(\cos\theta_\nu\ket{\nu_1}+\sin\theta_\nu\ket{\nu_2})|^2\\
=&|\cos^2\theta_\nu e^{-iE_1t}+\sin^2\theta_\nu e^{-iE_2t}|^2\\
=&|\cos^2\theta_\nu\cos(E_1t)-i\cos^2\theta_\nu\sin(E_1t)+\sin^2\theta_\nu\cos(E_2t)-i\sin^2\theta_\nu\sin(E_2t)|^2\\
=&\cos^4\theta_\nu\cos^2(E_1t)+\sin^4\theta_\nu\cos^2(E_2t)+2\sin^2\theta_\nu\cos^2\theta_\nu\cos(E_1t)\cos(E_2t)\\
&+\cos^4\theta_\nu\sin^2(E_1t)+\sin^4\theta_\nu\sin^2(E_2t)+2\sin^2\theta_\nu\cos^2\theta_\nu\sin(E_1t)\sin(E_2t)\\
=&\cos^4\theta_\nu+\sin^4\theta_\nu+2\sin^2\theta_\nu\cos^2\theta_\nu\cos[(E_2-E_1)t]\\
=&1-2\sin^2\theta_\nu\cos^2\theta_\nu\{1-\cos[(E_2-E_1)t]\}=1-4\sin^2\theta_\nu\cos^2\theta_\nu\sin^2\left[\frac{(E_2-E_1)}{2}t\right]\\
\approx&1-\sin^2(2\theta_\nu)\sin^2\left(\frac{\Delta m^2L}{4E}\right)\Rightarrow\pazocal{P}(\nu_e\to\nu_\mu)=\sin^2(2\theta_\nu)\sin^2\left(\frac{\Delta m^2L}{4E}\right)
\end{align*}
Current oscillation experiments measure:
\[
\Delta m_{12}^2=m_2^2-m_1^2\approx7.37\cdot10^{-5}\;\text{eV$^2$} \quad |\Delta m_{32}|^2=|m_3^2-m_2^2|^2\approx2.56\cdot10^{-3}\;\text{eV$^2$}
\]
with an ambiguity about the hierarchy still not resolved.\\
In the SM, there are three families, the neutrinos mixing matrix is a $3\times3$ with the same mathematical properties as the CKM one, it is called the \textbf{Pontecorvo-Maki-Nakagawa-Sakata matrix} (PMNS) and it connects flavour eigenstates with mass eigenstates.
\[
\begin{pmatrix}
    \nu_e\\ \nu_\mu\\ \nu_\tau
\end{pmatrix}=V_{\text{PMNS}}\begin{pmatrix}
    \nu_1\\ \nu_2\\ \nu_3
\end{pmatrix}
\]
\newpage
\section{The Standard Model}
\subsection{The Electroweak Theory}
Glashow, Salam and Weinberg provided the main ingredients for the unification of weak and EM interactions. Fundamental interactions are described by field theories, i.e. by a Lagrangian, invariant under appropriate symmetries which correspond to conservation laws of the theory. In the SM, EM and WI are related to the symmetry group SU(2)$\times$U(1). In mathematical terms, it has to be \textbf{renormalizable} which means that it must exist a mathematically consistent procedure that eliminates the infinities which arise in calculations of physical observables. As a consequence, the EW Lagrangian must not contain explicit mass terms so at the Lagrangian level both the gauge bosons and the fermions must be massless. The masses are then generated in the theory, without destroying the renormalizability, via the \textbf{Higgs mechanism}. This predicts the existence of one scalar, the Higgs boson $H$. The values of the fermion masses are left as free parameters but once they are fixed all the couplings are predicted by the theory.\\
The fundamental representation of SU(2)$\times$U(1) is given by three SU(2) and one U(1) gauge fields. In the SU(2) sector we have the \textbf{weak isospin} $I_W$ (here only isospin) and in the U(1) sector there is the \textbf{weak hypercharge} $Y_W$ (here only hypercharge). All the members of the same isospin multiplet have the same hypercharge, defined as $Y_W:=2(Q-I_{W3})$.\\
The triplet of fields corresponding to SU(2) is called $W=(W_1,W_2,W_3)$ with $I_W=1$ and $Y_W=0$, they interact with the weak isospin of the particles.\\
The field corresponding to U(1) is called $B$. It has isospin, electric charge and hypercharge equal to zero and it interacts with the weak hypercharge of the particle.\\
These four fields are \textbf{not} the physical fields which mediate the interactions: the CC interactions are mediated by $W^\pm$, linear combinations of $W_1$ and $W_2$, while for EM and NC interactions $Z$ and the photon field $A_\mu$ are linear combinations of $W_3$ and $B$.\\
The values of $I_W$ and $Y_W$ of the particles depends on the fact that the $W^\pm$ are coupled only to states with negative chirality. In each family of leptons, there are two left-handed leptons in a $I_W=\frac{1}{2}$ doublet. Unlike CC, NC also interact with the charged right-handed fermions but not with the right-handed neutrinos. The right-handed charged lepton of each family is an isospin singlet, $I_W=0$.\\
% The $W^\pm$ is universally coupled with the CKM-rotated states $d',s'$ and $b'$. Three isospin doublet times three colours equal to nine doublets, while there are 18 singlets in total. For NC, quark mixing is irrelevant.
\begin{table}[h]
    \centering
    \begin{tabular}{l|ccccc}
    \hline
    \cellcolor{gray!50} & \cellcolor{yellow!50}Spin & \cellcolor{yellow!50}$I_W$ & \cellcolor{yellow!50}$I_{W3}$ & \cellcolor{yellow!50}$Y_W$ & \cellcolor{yellow!50}$Q$\\
    \hline\hline
    \cellcolor{orange!50}$\nu_{lL}$ & 1/2 & 1/2 & +1/2 & $-$1 & 0 \\
    \hline
    \cellcolor{orange!50}$l^-_L$ & 1/2 & 1/2 & $-$1/2 & $-$1 & $-$1 \\
    \hline
    \cellcolor{orange!50}$l^-_R$ & 1/2 & 0 & 0 & $-$2 & $-$1 \\
    \hline
    \cellcolor{orange!50}$u_{L}$ & 1/2 & 1/2 & +1/2 & +1/3 & +2/3 \\
    \hline
    \cellcolor{orange!50}$d'_{L}$ & 1/2 & 1/2 & $-$1/2 & +1/3 & $-$1/3 \\
    \hline
    \cellcolor{orange!50}$u_{R}$ & 1/2 & 0 & 0 & +4/3 & +2/3 \\
    \hline
    \cellcolor{orange!50}$d_{R}$ & 1/2 & 0 & 0 & $-$2/3 & $-$1/3 \\
    \hline\hline
    \cellcolor{blue!50}$W^+$ & 1 & 1 & +1 & 0 & +1 \\
    \hline
    \cellcolor{blue!50}$W^-$ & 1 & 1 & $-$1 & 0 & $-$1 \\
    \hline
    \cellcolor{blue!50}$Z$ & 1 & 1,0 & 0 & 0 & 0 \\
    \hline
    \cellcolor{blue!50}$\gamma$ & 1 & 1,0 & 0 & 0 & 0 \\
    \hline\hline
    \cellcolor{purple!50}$H$ & 0 & 0 & 0 & 0 & 0\\
    \hline
    \end{tabular}
    \caption{Quantum numbers of the fields in the Standard Model}
    \label{aaa}
\end{table}\newline
$I_W$ and $Y_W$ are conserved in all known interactions. Left components of the spinors have $I_W\neq0$, they emit and absorb $W^\pm$ while right components have $I_W=0$ and they do not emit or absorb $W^\pm$. Both components have $Y_W\neq0$, they emit and absorb $Z$.\\
The field $W_\mu=(W_\mu^1,W_\mu^2,W_\mu^3)$ is a 4-vector in space-time and a vector in SU(2). The fields of the physical charged bosons are $W^\pm=\frac{1}{\sqrt{2}}(W_1\mp iW_2)$. For each doublet of fermions there is a 4-vector which represents the weak current $J_\mu:=(J_\mu^1,J_\mu^2,J_\mu^3)$. $W_\mu$ is coupled to $J_\mu$ through the dimensionless coupling constant $g$.\\
The field $B_\mu$ is a 4-vector in space-time and a scalar in SU(2) and it interacts with the neutral currents of the leptons through the coupling $g'$. The current generated by the hypercharge is twice the difference between the EM current and the neutral component of the NC.\\
Denote now with $A$ and $Z$ the physical fields that mediate, respectively, the EM and neutral currents. These are two orthogonal linear overlap of $W_3$ and $B$, which can be determined by requiring that the photon does not couple to the neutral particles. This transformation is given by a function of the two couplings $g$ and $g'$, i.e. of the rotation angle $\theta_w$ known as the \textbf{Weinberg angle}.
\beginbox{Mixing Angle $\theta_w$}
\[
\begin{pmatrix}
    Z\\A
\end{pmatrix}=\frac{1}{\sqrt{g^2+g'^2}}\begin{pmatrix}
    g & -g'\\g' & g
\end{pmatrix}
\begin{pmatrix}
    W_3\\B
\end{pmatrix}=\begin{pmatrix}
    \cos\theta_w & -\sin\theta_w\\ \sin\theta_w & \cos\theta_w
\end{pmatrix}
\begin{pmatrix}
    W_3\\B
\end{pmatrix}
\]
\endbox
The mixing angle is quite large, $\theta_w\approx29^\circ$. The interaction Lagrangian is an isoscalar:
\[
\pazocal{L}=g(J_\mu^1W_\mu^1+J_\mu^2W_\mu^2+J_\mu^3W_\mu^3)+\frac{1}{2}g'J_\mu^YB_\mu=\frac{1}{\sqrt{2}}(J_\mu^-W_\mu^++J_\mu^+W_\mu^-)+J_\mu^3(g_\mu^3+g'B_\mu)+g'J_\mu^{\text{EM}}B_\mu
\]
Introducing the neutral physical fields $A$ and $Z$, this becomes:
\[
\pazocal{L}=\underbrace{\frac{1}{\sqrt{2}}g(J_\mu^-W_\mu^++J_\mu^+W_\mu^-)}_{\text{CC}}+\underbrace{\frac{g}{\cos\theta_w}(J_\mu^3-\sin\theta_wJ_\mu^{\text{EM}})Z_\mu}_{\text{NC}}+\underbrace{g\sin\theta_wJ_\mu^{\text{EM}}A_\mu}_{\text{EM}}
\]
The constant which multiplies the last term has to be proportional to the electric charge, to ensure that the photon is not coupled to neutral particles, therefore:
\[
g\sin\theta_w=q_e=\sqrt{4\pi\alpha}
\]
All the interactions mediated by the four vector bosons are expressed in terms of two constants, the electric charge $q$ and the weak angle $\theta_w$. However, the model does not predict the values of the two fundamental constants, which must be determined experimentally. The relations between the fundamental constants, obtained from the low-energy value of $\alpha\approx\frac{1}{137}$ and the best measurement of $\theta_w$ ($\sin\theta_w\approx0.232$), are:
\beginbox{Coupling Constants}
\[
\frac{1}{\alpha}=\frac{4\pi}{g^2}+\frac{4\pi}{g'^2} \qquad \frac{4\pi}{g^2}=31.8 \qquad \frac{4\pi}{g'^2}=105.2
\]
\endbox
NC have important differences compared to CC. For example, there is no FCNC, i.e. fermions are only coupled with themselves. They do not have the simple V-A coupling $\gamma_\mu(1-\gamma_5)$ but are a mixture of both left and right couplings. In the SM, the 7 couplings $(g_L^{\nu_l}, g_L^l, g_R^l, g_L^u, g_R^u, g_L^d, g_R^d)$ are equal for the 3 families and are functions of only two parameters $\alpha_{\text{EM}}$ and $\theta_w$.\\
The experiments measure observables like cross sections or decay rates and compare calculated quantities to measured ones. This calculation is based on a perturbative series, where the lowest order is the tree-level and higher orders correspond to radiative corrections. 
\subsection{QCD}
The \textbf{colour} quantum number was introduced to avoid Pauli principle violation for the $\Delta^{++}$. Moreover, it is necessary to explain the value of $R$ in the $e^+e^-$ interactions which shows an excess of a factor 3. Similarly, it is necessary to explain the decay rate $\pi^0\to\gamma\gamma$. The theory must also explain confinement (no free quarks) and asymptotic freedom (quarks almost free at high $Q^2$).\\
Modern QCD is a gauge theory based on the symmetry SU(3)$_{\text{colour}}$, mathematically equivalent to SU(3)$_{\text{flavour}}$ but based on a different dynamics. The carriers of the force are 8 coloured massless spin-1 vector bosons called \textbf{gluons}.\\
The $\pi^0$ decay is an EM process but we discuss it here because it critically depends on the number of colours. Compute the decay amplitude by introducing an a-priori unknown normalization factor $N_c$:
\[
\Braket{\gamma\gamma|H_{\text{EM}}|\pi^0}\propto f_\pi\frac{N_c}{\sqrt{2}}\left(\frac{4}{9}e^2-\frac{1}{9}e^2\right)
\]
where $f_\pi$ is the decay constant of $\pi^0$, related to the wave-function overlap of the quarks and anti-quarks. The full computation, compared with the experimental measure, gives a value compatible with the QCD prediction $N_c=3$.\\
The colour is a charge, equivalent to the electric one, it is exactly conserved in strong interactions. Gluons carry two colours and are therefore self-coupled: vertices with three or four gluons are allowed in QCD, while for example in QED it happens only on higher orders with a triangle of fermions. The number of gluons comes from the Gell-Mann matrices, the generators of SU(3). The colour does not manifest directly in an observable property of the particle. The standard explanation of this fact requires that only \textit{white}, i.e. colour singlets, states can be physically existent. Quarks and gluons themselves cannot be observed as free states (\textbf{confinement}), they exist only as hadrons.\\
The study of the DIS shows that at high $Q^2$ the projectile \textit{sees} smaller objects inside the nucleon. At small distances, the force between quarks and gluons is apparently smaller and smaller, the quarks behave as free objects. Scattering onto free quarks is the origin of the Bjorken scaling. This effect is called \textbf{asymptotic freedom}. With increasing distance among the quarks (lower $Q^2$), the intensity of the strong force increases, keeping the quarks confined in the nucleon. On the other hand, at some distance the available energy becomes sufficient to create a new quark-anti-quark pair, leading to the production of new hadrons but preventing the emission of quarks as free particles. Mesons are colour singlets, a $q\overline{q}$ pair with symmetric wave-function while baryons are colour singlets $qqq$ with anti-symmetric wave-function.\\
A semi-classical approach tells us that for small distances, the potential has a Coulomb shape with a coupling $\alpha_{\text{S}}$. At large distances, it has a linearly increasing function responsible for the confinement.
\[
V(r)=\underbrace{-\frac{4\alpha_{\text{S}}}{3r}}_{\text{small $r$}}+\underbrace{kr}_{\text{large $r$}}
\]
Consider a two-body process at high $Q^2$, $q\overline{q}\to q\overline{q}$. There are 8 possible cases, according to $q\overline{q}g$ and it is basically impossible to disentangle on an event-by-event basis. A more effective approach for scattering processes is to reabsorb higher orders into an effective coupling $\alpha_{\text{S}}$. The loops increase with $Q^2$, but since the gluons are self-interacting there will be mainly gluon loops, hence $\alpha_{\text{S}}$ decreases with $Q^2$. This explains the nucleons and the DIS.
\newpage
\section{High Energy $\nu$ Interactions}
After 1960, the accelerator production of $\nu$-beams of high intensity and high energy has led to a dramatic development of our understanding of weak interactions. Neutrinos cross sections are very small, beams, detectors and experimental setups have to compensate for this. For example, statistically we need 30 billions $\nu$ of energy 1 GeV to get one interaction in 10 meters of iron.
\subsection{The $\nu$ Beam}
The relevant observable is the cross section and in order to measure it, the experiments need the flux of incoming $\nu$. A neutrino cannot be observed before its interaction, therefore the flux can only be computed statistically. The ingredients are:
\begin{enumerate}
    \item the inclusive differential cross sections of the $\pi^\pm$ and $K^\pm$ production in the target
    \item the collection and collimation of $\pi^\pm$ and $K^\pm$
    \item the distribution of the decay length and position
    \item the distribution of the $\nu$ decay angle
\end{enumerate}
Using all these elements, the flux is numerically computed, usually with MonteCarlo simulations and used in the analysis.\\
The statistical distribution of (1) and (2) can be directly measured, the momentum distribution of $\mu^\pm$ from $\pi^\pm/K^\pm$ decay can be computed and checked using their measurement in the decay and absorber tunnels. Collection and collimation (2) may use different strategies:
\begin{itemize}
    \item \textbf{wide band beam} (WBB): intense beam, not monochromatic
    \item \textbf{narrow band beam} (NBB): less intense but more monochromatic
\end{itemize}
In practice, both beams are optimized for different measurements.\\
As an example, the SPS at CERN accelerates $\sim5\cdot10^{13}$ protons per cycle to an energy\\
$E_p=450$\;GeV. The proton beam is extracted and sent to a target (copper, beryllium, graphite), the average secondary multiplicity is $\sim10$ charged, with energies from 10 to 100 GeV. The secondaries are then focused and let decay. The focusing process is a compromise between resolution (ideally monochromatic) and intensity (as many neutrinos as possible).\\
A good solution is the WBB beam where a Van der Meer horn selects with good acceptance $\pi^\pm$ and $K^\pm$. $\pi^\pm$ and $K^\pm$ decay in the decay tunnel. Again, the length of the tunnel is a compromise between cost and intensity, it should be about the average decay length. In the laboratory frame, the average $d_m$ is $d_m=\beta c\tau$ which, for $\pi^\pm$ of 50 GeV, is approximately 2800\,m. In reality, tunnels are only a few hundreds meters long. In these decays, only beams of $\nu_\mu$ can be created, electron neutrinos are small contaminations. Neutrinos cannot be directly measured, some information about their four-momentum comes from the kinematics of the decay $\pi^\pm/K^\pm\to\mu^\pm\nu_\mu$. $\pi^\pm/K^\pm$ have spin 0 and in the center of mass the decay is isotropic. Moreover, in the CM we have:
\[
\pi^\pm=(m_\pi,0,0) \quad \nu=(p^*,p^*\cos\theta^*, p^*\sin\theta^*) \quad \mu^\pm=(m_\pi-p^*,-p^*\cos\theta^*,-p^*\sin\theta^*)
\]
Kinematics tells us that:
\[
p^*=\frac{m_\pi^2-m_\mu^2}{2m_\pi} \quad E_\mu^*=m_\pi-p^*=\frac{m_\pi^2+m_\mu^2}{2m_\pi}
\]
The longitudinal momentum in the laboratory frame is obtained by a boost:
\[
p_\nu^\parallel\bigr|_{\substack{\text{LAB}}}=p\cos\theta=\gamma p^*\cos\theta^*+\beta\gamma p^*
\]
This has a minimum for $\cos\theta^*=-1$, corresponding to $p_\nu^\parallel\bigr|^{{\min}}_{\substack{\text{LAB}}}=\gamma p^*(\beta-1)\approx0$ and a maximum for $\cos\theta^*=1$:
\[
p_\nu^\parallel\bigr|^{{\max}}_{\substack{\text{LAB}}}=\gamma p^*(\beta+1)\approx2\gamma p^*=2\frac{E_\pi}{m_\pi}\frac{m_\pi^2-m_\mu^2}{2m_\pi}=E_\pi\left(1-\frac{m_\mu^2}{m_\pi^2}\right)=0.43E_\pi
\]
For $K^\pm$, the maximum is higher around 0.96$E_K$. The angular distribution of a $\nu$ with respect to a $\pi^\pm$ is:
\[
\frac{dn}{d\Omega}\approx\frac{1}{4\pi}\frac{4\gamma^2(1+\tan^2\theta)^{3/2}}{(1+\gamma^2\tan^2\theta)^2}
\]
The detectors must get only neutrinos and not the muons, pions and kaons. Therefore, a thick absorber is positioned at the end of the decay tunnel. At the SPS at CERN, this is made with 185m of iron + 220m of rock. The main contaminations for WBB $\nu_\mu$ come from $\overline{\nu}_\mu$ and vice-versa. For NBB, the relation between the radial distance of the impact point and the $\nu$ energy allows for a discrimination of the neutrino energy with a certain resolution and little $\pi/K$ ambiguity.
\subsection{CC $\nu$ Processes}
A very simple CC process is the pure lepton scattering $\nu_\mu e^-\to\mu^-\nu_e$. Kinematics tells us that:
\[
E_{\nu_\mu}^{\min}=\frac{(m_\mu+m_{\nu_e})^2-m_{\nu_\mu}^2-m_e^2}{2m_e}\approx\frac{m_\mu^2-m_e^2}{2m_e}\approx\frac{m_\mu^2}{2m_e}\approx11\;\text{GeV}
\]
The creation of $\mu$ requires high energy neutrinos. Repeating the same calculations with $\tau$ gives a minimum energy of roughly 3 TeV, not possible with present accelerators. The cross-section of this process is:
\[
\sigma(\nu_\mu e^-\to\mu^-\nu_e)=\frac{2G_F^2m_eE_\nu}{\pi}\left(1-\frac{m_\mu^2}{2m_eE_\nu}\right)^2
\]
However, the pure lepton process is so rare that it is hard to get statistical significance. More common processes are quasi-elastic scatterings:
\[
\nu_\mu n\to\mu^-p \quad \overline{\nu}_\mu p\to\mu^+n
\]
In Fermi theory, with $p$ and $n$ point-like and a threshold $s=2m_NE_\nu$, we get $\sigma\propto2G_F^2m_NE_\nu$:
\[
\frac{\sigma(\nu\to N)}{\sigma(\nu\to e)}=\frac{m_N}{m_E}\approx2000
\]
In the V-A theory, $f_V(Q^2)$ and $f_A(Q^2)$ makes $\sigma$ not dependent on $E_\nu$. At high $Q^2$, as expected, the nucleons \textit{break} and the interactions \textit{see} quarks instead of nucleons: quasi-elastic essentially disappears. At high $Q^2$, the quark model is expected to describe well the neutrinos interactions. 
\subsection{NC $\nu$ Processes}
The search for NC events began in the early '60s, when the EW theory was still thought to be not renormalizable. The searches were limited by FCNC, decays like $K^+\to\pi^+e^+e^-$ and $K^0\to\mu^+\mu^-$ were searched but not found. The only escape from this difficulty is to make use of neutral particles which are not affected by EW interactions, the neutrinos. The signature for this process is given by the absence in the final state of a charged lepton, which is unavoidable in the CC coupling. The events were of the type:
\[
\begin{aligned}
&\nu_\mu+N\to\nu_\mu+X &&\overline{\nu}_\mu+N\to\overline{\nu}_\mu+X\\
&\nu_\mu+e^-\to\nu_\mu+e^- &&\overline{\nu}_\mu+e^-\to\overline{\nu}_\mu+e^-
\end{aligned}
\]
where $X$ is an hadronic system without leptons. Events are defined as NC if no $\mu^\pm$ is present and if no charged track escapes the fiducial volume. Instead, events are CC if a clearly visible $\mu^\pm$ is present and the $\mu^\pm$ has to exit out of the chamber. For $\nu$ beams, there were 102 NC and 428 CC while for $\overline{\nu}$ beam they had 64 NC and 148 CC, results consistent with the presence of NC.\\
To measure these processes, the strategy is the same as in the CC case, with a simplification due to the absence of FCNC.
\[
R_\nu:=\frac{\sigma_{\text{NC}}(\nu N)}{\sigma_{\text{CC}}(\nu N)}\approx\frac{1}{2}-\sin^2\theta_w+\frac{20}{27}\sin^4\theta_w \quad R_{\overline{\nu}}:=\frac{\sigma_{\text{NC}}(\overline{\nu} N)}{\sigma_{\text{CC}}(\overline{\nu} N)}\approx\frac{1}{2}-\sin^2\theta_w+\frac{20}{9}\sin^4\theta_w
\]
These values are well-defined and, in principle, easy to measure. CC are easy to detect due to the presence of $\mu^\pm$ and relatively background free. NC are hardly distinguishable from cosmics and low-energy CC. Most recent results show:
\[
\sin^2\theta_w=
\left\{
\begin{aligned}
&0.2356\pm0.0050 \quad \text{CHARM}\\
&0.2250\pm0.0050 \quad \text{CDHS}\\
&0.2332\pm0.0015 \quad \text{today's best from PDG}\\
&0.2251\pm0.0039 \quad \text{today's best from PDG}
\end{aligned}
\right.
\]
The last two results differ because of some definitions of higher order parameters.\\
Due to the presence of a charged $\mu^\pm$, CC events are easy to detect and relatively background free. NC instead are hardly distinguishable from cosmics and low-energy CC.\\
The cleanest NC leptonic processes are $\nu_\mu e^-\to\nu_\mu e^-$ and $\overline{\nu}_\mu e^-\to\overline{\nu}_\mu e^-$.\\
The problem with these processes is that the cross section is very small and we need to select the tiny number of signal events from the overwhelming NC hadronic events. The extraction of the signal requires the rejection of the background. The main one is due to NC hadronic interactions, without $\mu^\pm$ in the final state, with one or more $\pi^0$: the photons due to $\pi^0$ decays mimic the EM shower. To reject those events, the deposit of energy in the early scintillators is used. Since $\pi^0\to2\gamma\to4e^\pm$, a scintillator sees 4 minimum ionizing particles instead of one. In this way, by using only the part of the detector immediately upstream of the scintillator, a much better isolation of the signal is obtained, at the price of a reduced statistics. 
\[
R_{\text{NC}}^{\nu_\mu e}=\frac{\sigma_{\text{NC}}(\nu_\mu e^-)}{\sigma_{\text{NC}}(\overline{\nu}_\mu e^-)}=3\frac{1-4\sin^2\theta_w+\frac{16}{3}\sin^4\theta_w}{1-4\sin^2\theta_w+16\sin^4\theta_w}
\]
Results from $\nu_\mu e$ are:
\[
\sin^2\theta_w=
\left\{
\begin{aligned}
&0.2324\pm0.0058\pm0.0059 \quad \text{CHARM}\\
&0.2311\pm0.0077 \quad \text{today's best from PDG}\\
&0.2230\pm0.0077 \quad \text{today's best from PDG}
\end{aligned}
\right.
\]
\end{document}